\begin{latexonly}
\newpage
\end{latexonly}
\subsection{sddsoutlier}
\label{sddsoutlier}

\begin{itemize}
\item {\bf description:}
{\tt sddsoutlier} does outlier elimination of rows from SDDS tabular
data.  An ``outlier'' is a data point that is statistically unlikely
or else invalid.
\item {\bf example:}
Eliminate ``bad'' beam-position-monitor readouts from PAR x BPM data, where
a bad readout is one that is more than three standard deviations from the
mean:
\begin{flushleft}{\tt
sddsoutlier par.bpm par.bpm1 -columns=P?P?x -stDevLimit=3
}\end{flushleft}
Fit a line to readout P1P1x vs P1P2x, then eliminate points too far from the line.
\begin{flushleft}{\tt
sddspfit par.bpm -pipe=out -columns=P1P2x,P1P1x \\
| sddsoutlier -pipe=in par.2bpms -column=P1P1xResidual -stDevLimit=2
}\end{flushleft}
Same, but refit and redo outlier elimination based on the improved fit:
\begin{flushleft}{\tt
sddspfit par.bpm -pipe=out -columns=P1P2x,P1P1x \\
| sddsoutlier -pipe par.2bpms -column=P1P1xResidual -stDevLimit=2 \\
| sddspfit -pipe -columns=P1P2x,P1P1x \\
| sddsoutlier -pipe=in par.2bpms -column=P1P1xResidual -stDevLimit=2 
}\end{flushleft}

\item {\bf synopsis:} 
\begin{flushleft}{\tt
sddsoutlier [-pipe=[input][,output]] [{\em inputFile}] [{\em outputFile}]
[-columns={\em listOfNames}] [-excludeColumns={\em listOfNames}]
[-stDevLimit={\em value}] [-absLimit={\em value}] [-absDeviationLimit={\em value}]
[-minimumLimit={\em value}] [-maximumLImit={\em value}] 
[-chanceLimit={\em value}] [-invert] [-verbose] [-noWarnings] 
[\{-markOnly | -replaceOnly=\{lastValue | nextValue | interpolatedValue | value={\em number}\}\}]
}\end{flushleft}
\item {\bf files:}
{\em inputFile} contains column data that is to be winnowed using outlier elimination.
If {\em inputFile} contains multiple pages, the are treated separately.  {\em
outputFile} contains all of the array and parameter data, but only those rows of the
tabular data that pass the outlier elimination.  {\em Warning}: if {\em outputFile} is
not given and {\tt -pipe=output} is not specified, then {\em inputFile} will be
overwritten.
\item {\bf switches:}
    \begin{itemize}
    \item {\tt -pipe[=input][,output]} --- The standard SDDS Toolkit pipe option.
    \item {\tt -columns={\em listOfNames}} --- Specifies a comma-separated list of
        optionally wildcard containing column names.  Outlier analysis and elimination
        will be applied to the data in each of the specified columns independently.
        No row that is eliminated by outlier analysis of any of these columns will      
        appear in the output.  If this option is not given, all columns are 
        included in the analysis.
    \item {\tt -excludeColumns={\em listOfNames}} --- Specifies a comma-separated list of
        optionally wildcard containing column names that are to be excluded from 
        outlier analysis.
    \item {\tt -stDevLimit={\em value}} --- Specifies the number of standard deviations
        by which a data point from a column may deviate from the average for the column
        before being considered an outlier.
    \item {\tt -absLimit={\em value}} --- Specifies the maximum absolute value that
        a data point from a column may have before being considered an outlier.
    \item {\tt -absDeviationLimit={\em value}} --- Specifies the maximum absolute value
        by which a data point from a column may deviate from the average for the column
        before being considered an outlier.
    \item {\tt -minimumLimit={\em value}}, {\tt -minimumLimit={\em value}} --- 
        Specify minimum or maximum values that data points may have without being considered
        outliers.
    \item {\tt -chanceLimit={\em value}} --- Specifies placing a lower limit on the probability
        of seeing a data point as a means of removing outliers.  Gaussian statistics are
        used to determine the probability that each point would be seen in sampling a gaussian
        distribution a given number of times (equal to the number of points in each page).
        If this probability is less than {\em value}, then the point is considered an outlier.
        Using a larger {\em value} results in elimination of more points.
    \item {\tt -invert} --- Specifies that only outlier points should be kept.
    \item {\tt -markOnly} --- Specifies that instead of deleting outlier points, they
        should be only marked as outliers.  This is done by creating a new column
        ({\tt IsOutlier}) in the output file that contains a 1 (0) if the row has (no)
        outliers.  If {\tt IsOutlier} is in the input file, rows with a value of 1 are
        treated as outliers and essentially ignored in processing.  Hence, successive
        invocations of {\tt sddsoutlier} in a data-processing pipeline make use of 
        results from previous invocations even if {\tt -markOnly} is given.  Note: if {\tt -markOnly}
        is not given, then the presence of {\tt IsOutlier} in the input file has no effect.
    \item {-tt -replaceOnly=\{lastValue | nextValue | interpolatedValue | value={\em number}\}} ---
        Specifies replacing outliers rather than removing them.  {\tt lastValue} ({\tt nextValue})
        specifies replacing with the previous (next) value in the column.  
        {\tt interpolatedValue} specifies interpolating a new value from the last and next value
        (with row number as the independent quantity).  {\tt value={\em number}} specifies 
        replacing outliers with {\em number}. 
    \item {\tt -verbose} --- Specifies that informational printouts should be provided.
    \item {\tt -noWarnings} --- Specifies that warnings should be suppressed.
    \end{itemize}
\item {\bf see also:}
    \begin{itemize}
    \item \hyperref{Data for Examples}{Data for Examples (see }{)}{exampleData}
    \item \progref{sddspfit}
    \item \progref{sddsgfit}
    \item \progref{sddsexpfit}
    \item \progref{sddscorrelate}
    \end{itemize}
\item {\bf author:} M. Borland, ANL/APS.
\end{itemize}



