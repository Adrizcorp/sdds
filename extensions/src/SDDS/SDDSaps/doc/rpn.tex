\begin{latexonly}
\newpage
\end{latexonly}
\subsection{rpn Calculator Module}
\label{rpn}
\label{rpn Calculator Module}

\begin{itemize}
\item {\bf description:} \hspace*{1mm}\\

Many of the SDDS toolkit programs employ a common Reverse Polish Notation (RPN) calculator module for equation evaluation.
This module is based on the {\tt rpn} programmable calculator program.  It is also available in a commandline
version called {\tt rpnl} for use in shell scripts.  This manual page discusses the programs {\tt rpn} and 
{\tt rpnl},  and indicates how the {\tt rpn} expression evaluator is used in SDDS tools.
\item {\bf examples:} \\
Do some floating-point math using shell variables:
(Note that the asterisk (for multiplication) is escaped in order to protect it from interpretation by the shell.)
\begin{flushleft}{\tt
set pi = 3.141592\\
set radius = 0.15\\
set area = `rpnl \$pi \$radius 2 pow \verb|\|*`\\
}\end{flushleft}
Use {\tt rpn} to do the same calculation:
\begin{flushleft}{\tt
% rpn\\
rpn> 3.141592 sto pi\\
rpn> 0.15 sto radius\\
rpn> radius 2 pow pi *\\
	      0.070685820000000\\
rpn> quit\\
% \\
}\end{flushleft}
\item {\bf synopsis:}
\begin{flushleft}{\tt
rpn [{\em filenames}]\\
rpnl {\em rpnExpression}
}\end{flushleft}
\item {\bf Overview of {\tt rpn} and {\tt rpnl}}:

{\tt rpn} is a program that places the user in a Reverse Polish Notation calculator shell.  Commands to {\tt rpn} consist
of generally of expressions in terms of built-in functions, user-defined variables, and user-defined functions.  Built-in
functions include mathematical operations, logic operations, string operations, and file operations.  User-defined
functions and variables may be defined ``on the fly'' or via files containing {\tt rpn} commands. 

The command {\tt rpn} {\em filename} invokes the {\tt rpn} shell with {\em filename} as a initial command file.
Typically, this file would contain instructions for a computation.  Prior to execution of any files named the
commandline, {\tt rpn} first executes the instructions in the file named by the environment variable {\tt RPN\_DEFNS}, if
it is defined.  This file can be used to store commonly-used variable and function definitions in order to customize the
{\tt rpn} shell.  This same file is read by {\tt rpnl} and all of the SDDS toolkit programs that use the {\tt rpn}
calculator module.  An example of such a file is included with the code.

As with any RPN system, {\tt rpn} uses stacks. Separate stacks are maintained for numerical, logical, string data, and
command files.

{\tt rpnl} is essentially equivalent to executing {\tt rpn}, typing a single command, then exiting.  However, {\tt rpnl}
has the advantage that it evaluates the command and prints the result to the screen without any need for user input.
Thus, it can be used to provide floating point arithmetic in shell scripts.  Because of the wide variety of operations
supported by the {\tt rpn} module and the availability of user-defined functions, this is a very powerful feature even for
command shells that include floating point arithmetic.

Built-in commands may be divided into four broad categories: mathematical operations, logical operations, string
operations, and file operations.  (There are also a few specialized commands such as creating and listing user-defined
functions and variables; these will be addressed in the next section).  Any of these commands may be characterized
by the number of items it uses from and places on the various stacks.

\begin{itemize}
\item Mathematical operations:
\begin{itemize}
\item Using {\tt rpn} variables:\\
The {\tt sto} (store) function allows both the creation of {\tt rpn} variables and modification of their
contents.  {\tt rpn} variables hold double-precision values.  The variable name may be any string starting
with an alphabetic character and containing no whitespace.  The name may not be one used for a built-in
or user-defined function.   There is no limit to the number of variables that may be defined.

For example, {\tt 1 sto one} would create a variable called {\tt one} and store the value 1 in it.  To recall the
value, one simply uses the variable name.  E.g., one could enter {\tt 3.1415925 sto pi} and later enter
{\tt pi } to retrieve the value of $\pi$.

\item Basic arithmetic: {\tt + - * / sum}\\

With the exception of {\tt sum}, these operations all take two values
from the numeric stack and push one result onto the numeric stack.
For example, {\tt 5 2 -} would push 5 onto the stack, push 2 onto the
stack, then push the result (3) onto the stack.  

{\tt sum} is used to sum the top {\tt n} items on the stack, exclusive
of the top of the stack, which gives the number of items to sum.  For
example, {\tt 2 4 6 8 4 sum} would put the value {\tt 20} on the stack.

\item Basic scientific functions: {\tt sin cos acos asin atan atan2 sqrt sqr pow exp ln}\\

With the exception of {\tt atan2} and {\tt pow}, these operations all take one item from the numeric stack and push one
result onto that stack.

{\tt sin} and {\tt cos} are the sine and cosine functions, while {\tt asin}, {\tt acos}, and {\tt atan} are inverse
trigonometic functins.  {\tt atan2} is the two-argument inverse tangent: {\tt x y atan2} pushes the value ${\rm atan(y/x)}$
with the result being in the interval ${\rm [-\pi, \pi]}$.

{\tt sqrt} returns the positive square-root of nonnegative values.  {\tt sqr} returns the square of a value.  {\tt pow}
returns a general power of a number: {\tt x y pow} pushes ${\rm x t}$ onto the stack.  Note that if {\tt y} is
nonintegral, then {\tt x} must be nonnegative.

{\tt exp} and {\tt ln} are the base-e exponential and logarithm functions.

\item Special functions: {\tt Jn Yn cei1 cei2 erf erfc gamP gamQ lngam}\\
{\tt Jn} and {\tt Yn} are the Bessel functions of integer order of the first and second kind\cite{Abramowitz}.  Both take
two items from the stack and place one result on the stack.  For example, {\tt x i Jn} would push ${\rm J_i(x)}$ onto the
stack.  Note that ${\rm Y_n(x)}$ is singular at x=0.

{\tt cei1} and {\tt cei2} are the 1st and 2nd complete elliptic integrals.  The argument is the modulus k, as seen in
the following equations (the functions K and E are those used by Abramowitz\cite{Abramowitz}).
\[{\rm cei1(k) = K(k^2) = \int_0^{\pi/2} \frac{d\theta}{\sqrt{1 - k^2 sin^2 \theta}}  } \] \\
\[{\rm cei2(k) = E(k^2) = \int_0^{\pi/2} \sqrt{1 - k^2 sin^2 \theta} d\theta }\]

{\tt erf} and {\tt erfc} are the error function and complementary error function.  By definition, ${\rm erf(x) + erfc(x)}$
is unity.  However, for large x, {\tt x erf 1 -} will return 0 while {\tt x erfc} will return a small, nonzero value.
The error function is defined as\cite{Abramowitz}:
\[ {\rm erf(x) = \frac{2}{\sqrt{\pi}} \int_0 x e^{-t^2} dt} \]
Note that ${\rm erf(x/\sqrt{2}) }$ is the area under the normal Gaussian curve between ${\rm -x}$ and ${\rm x}$.

{\tt gamP} and {\tt gamQ} are, respectively, the incomplete gamma function and its complement
\cite{Abramowitz}:
\[{\rm gamP(a, x) = 1-gamQ(a, x) = \frac{1}{\Gamma(a)} \int_0^x e^{-t} t^{a-1} dt \hspace*{10mm}  a>0} \]
These functions take two arguments; the 'a' argument is place on the stack first.

{\tt lngam} is the natural log of the gamma function.  For integer arguments, {\tt x lngam} is 
${\rm ln( (x-1)!) }$.  The gamma function is defined as\cite{Abramowitz}:
\[ {\rm \Gamma (x) = \int_0^\infty t^{x-1} e^{-t} dt } \]

\item Numeric stack operations: {\tt cle n= pop rdn rup stlv swap view ==}\\
{\tt cle} clears the entire stack, while {\tt pop} simply removes the top element.  {\tt ==} duplicates the top item on
the stack, while {\tt x n=} duplicates the top x items of the stack (excluding the top itself).  {\tt swap} swaps the top
two items on the stack. {\tt rdn} (rotate down) and {\tt rup} (rotate down) are stack rotation commands, and are the
inverse of one another.  {\tt stlv} pushes the stack level (i.e., the number of items on the stack) onto the stack.
Finally, {\tt view} prints the entire stack from top to bottom.

\item Random number generators: {\tt rnd grnd}\\
{\tt rnd} returns a random number from a uniform distribution on ${\rm [0, 1]}$.  {\tt grnd} returns
a random number from a normal Gaussian distribution.

\item Array operations: {\tt mal [ ]}\\
{\tt mal} is the Memory ALlocation command; it pops a single value from the numeric stack, and returns a
``pointer'' to memory sufficient to store the number of double-precision values specified by that value.  This
pointer is really just an integer, which can be stored in a variable like any other number.  It is used to place
values in and retrieve values from the allocated memory.  

\verb|]| is the memory store operator. A sequence of the form {\tt {\em value} {\em index} {\em addr} ]} results in
{\em value} being stored in the {\em index} position of address {\em addr}.  {\em value}, {\em index}, and {\em
addr} are consumed in this operation.  Indices start from 0.

Similarly, {\tt {\em index} {\em addr} [} value pushes the value in the {\em index} position of address {\em addr}
onto the stack.  {\em index} and {\em addr} are consumed in this operation.

\item Miscellaneous: {\tt tsci int}\\
{\tt tsci} allows one to toggle display between scientific and variable-format notation.  In the former, all
numbers are displayed in scientific notation, whereas in the later, only sufficiently large or small numbers are
so displayed.  (See also the {\tt format} command below.)

{\tt int} returns the integer part of the top value on the stack by truncating the noninteger part.  

\end{itemize}
\item {\bf Logical operations}: \verb.! && < == > ? $ vlog ||.\\
\begin{itemize}
\item Conditional execution: {\tt ?}
The question-mark operator functions to allow program branching.  It is meant to remind the user of the C operator for
conditional evaluation of expressions.  A conditional statement has the form\\
{\tt ? {\em executeIfTrue} : {\em executeIfFalse} \$}\\
The colon and dollar sign function as delimiters for the conditionally-executed instructions.  The {\tt ?} operator pops the
first value from the logic stack.  It branches to the first set of instructions if this value is ``true'', and to the
second if it is ``false''.

\item Comparisons: {\tt < == > }\\
These operations compare two values from the numeric stack and push a value onto the logic stack indicating the result.
Note that the values from the numeric stack are left intact.  That is, these operations push the numeric values back onto the
stack after the comparison.
\item Logic operators: \verb. && || !.\\
These operators consume values from the logic stack and push new results onto that stack.
\verb|&&| returns the logical and of the top two values, while {\tt ||} returns the logical or.
{\tt !} is the logical negation operator.
\item Miscellaneous: {\tt vlog}\\
This operator allows viewing the logic stack.  It lists the values on the stack starting at the top.

\item Examples:\\
Suppose that a quantity is tested for its sign.  If the sign is negative, then have the conditional return a -1, if the sign is positive then return a +1.

Suppose we are running in the rpn shell and that the quantity 4 is initially pushed onto the stack.  The command ``\verb.0 < ? -1 : 1 $.\ '' that accomplishes the sign test will be executed as follows.
\begin{flushleft}
\begin{verbatim}
command     stack    logical stack
0           0        stack empty
            4

<           0        false  <-- new value
            4

? 1 : -1 $  1        stack empty
            0  
            4
\end{verbatim}
\end{flushleft}
In order to keep the stack small, the command should be written ``\verb.0 < pop pop ? -1 : 1 $.\ '', where the pop commands would eliminate the 0 and 4 from the stack before the conditional is executed.

If the command is executed with rpnl command in a C-shell, then the \$
character has to be followed by a blank space to prevent the shell
from interperting the \$ as part of variable:\\

C-shell\verb.>.\ rpnl \verb."4 0 < pop pop ? -1 : 1 $ ".\\

If the command is executed in a C-shell sddsprocess command to create
a new column, then we write:\\
\begin{flushleft}
\begin{verbatim}
sddsprocess <infile> <outfile> \
   -def=col,NewColumn,"OldColumn 0 < pop pop ? -1 : 1 $ "

\end{verbatim}
\end{flushleft}
which is similar to the rpnl command above.

If the sddsprocess command is run in a tcl/tk shell, the \$ character can be escaped with a backslash as well as with a blank space:\\
\begin{flushleft}
\begin{verbatim}
sddsprocess <infile> <outfile> \
   "-def=col,NewColumn,OldColumn 0 < pop pop ? -1 : 1 \$"

\end{verbatim}
\end{flushleft}
Note that the double quotes enclose the whole command argument, not just the sub-argument.

\end{itemize}
\item {\bf String operations}: {\tt "" =str cshs format getformat pops scan sprf vstr xstr}\\

\begin{itemize}
\item Stack operations: {\tt "" =str pops vstr}\\
To place a string on the string stack, one simply encloses it in double quotation marks.
{\tt =str} duplicates the top of the string stack.  {\tt pops} pops the top item off
of the string stack.  {\tt vstr} prints (views) the string stack, starting at the top.
\item Format operations: {\tt format getformat}\\
{\tt format} consumes the top item of the string stack, and causes it to be used as
the default printf-style format string for printing numbers.  {\tt getformat} 
pushes onto the string stack the default printf-style format string for printing numbers.
\item Print/scan operations: {\tt scan sprf}\\
{\tt scan} consumes the top item of the string stack and scans it for a number; it
pushes the number scanned onto the string stack, pushes the remainder of the string
onto the string stack, and pushes true/false onto the logic stack to indicate
success/failure.  {\tt sprf} consumes the top of the string stack to get a sprintf
format string, which it uses to print the top of the numeric stack; the resulting
string is pushed onto the string stack.  The numeric stack is left unchanged.
\item Other operations: {\tt cshs xstr}\\
{\tt cshs} executes the top string of the stack in a C-shell subprocess; note that
if the command requires terminal input, {\tt rpn} will hang.  {\tt xstr} executes
the top string of the stack as an rpn command.
\end{itemize}

\item {\bf File operations}: {\tt @ clos fprf gets open puts}\\
\begin{itemize}
\item Command file input: {\tt @}\\
The {\tt @} operator consumes the top item of the string stack,
pushing it onto the command file stack.  The command file is executed
following completion of processing of the current input line.  Command
file execution may be nested, since the files are on a stack.  The
name of the command file may have options appended to it in the format
{\tt {\em filename},{\em option}}.  Presently, the only option
recognized is 's', for silent execution.  If not present, the command file
is echoed to the screen as it is executed.  

Example: {\tt "commands.rpn,s" @} would silently execute the {\tt rpn}
commands in the file {\tt commands.rpn}.

\item Opening and closing files: {\tt clos open}\\
{\tt open} consumes the top of the string stack, and opens a file with
the name given in that element.  The string is of the format {\tt {\em
filename},{\em option}}, where {\em option} is either 'w' or 'r' for
write or read.  {\tt open} pushes a file number onto the numeric
stack.  This should be stored in a variable for use with other file IO
commands.  The file numbers 0 and 1 are predefined, respectively, as
the standard input and standard output.

{\tt clos} consumes the top of the numeric stack, and uses it as the
number of a file to close.  

\item Input/output commands: {\tt fprf gets puts}\\
These commands are like the C routines with similar names.
{\tt fprf} is like fprintf; it consumes the top of the string stack
to get a fprintf format string for printing a number.  It consumes
the top of the numeric stack to get the file number, and uses the
next item on the numeric stack as the number to print.  This number
is left on the stack.

{\tt gets} consumes the top of the numeric stack to get a file
number from which to read.  It reads a line of input from the
given file, and pushes it onto the string stack.  The trailing
newline is removed.  If successful, {\tt gets} pushes true onto
the logic stack, otherwise it pushes false.

{\tt puts} consumes the top of the string stack to get a string to
output, and the top of the numeric stack to get a file number.  Unlike
the C routine of the same name, a newline is {\em not} generated.
Both {\tt puts} and {\tt fprf} accept C-style escape sequences for
including newlines and other such characters.

\end{itemize}

\item {\bf author:} M. Borland, ANL/APS.
\end{itemize}
