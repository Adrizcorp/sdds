% $Log: sddsplot.tex,v $
% Revision 1.19  2010/06/22 15:52:02  soliday
% Fixed a problem with the -replicate option description.
%
% Revision 1.18  2010/06/03 16:37:27  borland
% Added documentation for -graph fixForFile, fixForName, and fixForRequest.
%
% Revision 1.17  2009/07/16 17:42:24  borland
% Added documentation of the fractionalDeviation qualifer to -mode.
% Added documentation of -replicate.
%
% Revision 1.16  2009/03/03 17:45:46  borland
% Updated to include autolog mode.
%
% Revision 1.15  2006/05/22 22:20:48  jiaox
% Added usage for user defined linetype and dashed lines.
%
% Revision 1.14  2005/08/23 13:36:11  borland
% Added documentation for the fill qualifier of the -graphic option.
%
% Revision 1.13  2005/03/16 16:52:04  shang
% update the document for plotting with logscale
%
% Revision 1.12  2004/03/26 21:10:34  shang
% added documentation of -xExclude, -yExclude, -toPage and -fromPage options
%
% Revision 1.11  2003/11/14 16:10:41  soliday
% Fixed a font problem.
%
% Revision 1.10  2003/04/08 22:11:04  shang
% added missing curly bracket for timeFilter option doc
%
% Revision 1.9  2003/04/07 16:04:12  shang
% added document for -timeFilter option
%
% Revision 1.8  2002/07/17 14:49:13  soliday
% Added coffset documentation.
%
% Revision 1.7  2000/03/14 19:38:57  soliday
% Changes made by Borland.
%
% Revision 1.6  2000/01/10 19:04:54  borland
% Updated documentation.
%
% Revision 1.5  1998/04/29 23:14:05  emery
% Changed \PSFigure command to use the graphicx package.
%
% Revision 1.4  1998/04/17 22:12:39  emery
% Using \newcommand{\PSFigure} command again now
% that epsf.sty is found again.
%
% Revision 1.3  1996/05/31  21:04:35  emery
% Added \PSFigure command for charSet figure
%
\begin{latexonly}
\newpage
\end{latexonly}

\subsection{sddsplot}
\label{sddsplot}

\begin{itemize}
\item {\bf description:}
{\tt sddsplot} is a general purpose device-independent graphics program for displaying parameter and column data
from SDDS files.  The program is equally capable of quick-and-dirty plots and publication quality graphics.  It
allows organization of large amounts of data from multiple files into useful plots with minimal effort.  It
provides line, point, symbol, impulse, error-bar, and arrow plotting, with automatic variation of color, linetype,
etc.  It can do data winnowing using the data to be graphed or other data in the file.  Parameters from a file can
be designated for use as plot labels, legends, or for placement on the plot in specified locations.  Data pages may
be tagged and sorted by multiple criteria.

{\tt sddsplot} supports various flavors of Postscript, various windows options, and numerous graphics terminals.  For
X-windows, a GUI interface is generated that supports zoom/pan, cursor readout, movie mode, and much more.

\item {\bf examples:} 
Plot the horizontal beta function for the APS design:
\begin{flushleft}{\tt 
sddsplot -columnNames=s,betax APS0.twi
}
\end{flushleft}
Plot the Twiss functions for the APS design, using different line types for
each quantity:
\begin{flushleft}{\tt 
sddsplot -columnNames=s,'(beta?,etax)' APS0.twi -graph=line,vary 
}
\end{flushleft}
Plot the Twiss functions for APS lattices, one plotting page per lattice (i.e., per
data page), with different linetypes and a legend:
\begin{flushleft}{\tt 
sddsplot -columnNames=s,'(beta?,etax)' APS.twi -graphic=line,vary -legend
        -split=page -separate=page
}\end{flushleft}
Plot the Twiss functions for APS lattices, one plotting page per function, with each 
data page shown with a different line type:
\begin{flushleft}{\tt 
sddsplot -columnNames=s,'(beta?,etax)' APS.twi -graphic=line,vary
        -split=page -groupby=nameIndex -separate=nameIndex
}
\end{flushleft}
Plot the Twiss functions for the APS design, using a common scale for the beta functions
and another for eta:
\begin{flushleft}{\tt
sddsplot -graphic=line,vary APS0.twi -columnNames=s,beta? -yScalesGroup=id=beta 
-columnNames=s,etax -yScalesGroup=id=eta
}

\end{flushleft}

\item {\bf sddsplot concepts:}\\

{\tt sddsplot} has a very large number of options and is very flexible.   In most cases, only a very few
of these options are employed.  In order to make best use of {\tt sddsplot}, it helps to be familiar with
certain concepts. 

{\tt sddsplot} supports multiple ``plot pages'' and multiple ``panels'' per page.  In this context, a
``plot page'' is a separate sheet of paper for hardcopy devices, and the equivalent for interactive
devices.  For example, when using the X-windows interface just described, separate plot pages are held in
memory so that the user may go back and forth between them rapidly, or run them as a movie.  A plot page
may contain several nonoverlapping panels, each displaying essentially independent graphics.  Presently,
{\tt sddsplot} divides the plot page into an array of plot panels, each of equal size.  The default is one
plot panel per plot page.

Within each plot panel, {\tt sddsplot} may display data from any number of ``plot requests''.  A plot
request is a specification to {\tt sddsplot} of what data to plot from what files, and how to do it.  A plot
request must contain or indicate a list of names of columns and parameters to display, as well as the names
of one or more files from which to extract the data.  The data from plot requests are organized into plot
panels and plot pages according to certain defaults or explicit instructions.  One frequent choice is to
move to a new panel for each plot request.  However, one may also regroup data to display data from
different plot requests together.

For each request, the names of columns and parameters are grouped to form sets of sets of data
element names.  For example, {\tt -columnNames=s,(betax,betay,etax)} results in formation of three
sets of pairs: {\tt (s, betax)}, {\tt (s, betay)}, and {\tt (s, etax)}.  In a more complicated
example, the sets of dataname sets might include names of error-bar data (e.g., {\tt (x, y, ySigma)})
or vector components (e.g., {\tt (x, y, Ex, Ey)}).  To avoid confusion, a set of datanames like those
just listed will be referred to as a ``name group''.  Each name group for a request is given a
sequential ``name index'', which can be used as shown in the last example above.

Each panel is divided into two regions, a ``plot space'' (or ``pspace'') and a ``label space''.  The
pspace is the region where data is displayed.  Outside the pspace is the label space, where labels
and legends normally appear.

Any point on any plot panel can be referenced by unit coordinates that start at zero in the lower
left corner of the panel and end at unity in the upper right corner.  The extent of the pspace is
given in these coordinates.  By default, the region the pspace occupies in these coordinates is
[0.15, 0.90]x[0.15, 0.90].  The extent of the pspace may be changed explicitly, or it
may be altered implicitly by certain switches (e.g., to make room for legends).  The data or user's
coordinates, referred to as (x, y), are mapped onto this space, as are the ``pspace coordinates'',
(p, q).  The latter are [0, 1]x[0, 1] coordinates.

When {\tt sddsplot} reads data in from files, it collects it into internal data sets. By default,
each of these internal data sets contains the all of the data for one name group from one file.  That
is, an internal data set normally contains all of the data for a name group from an SDDS data set.
The phrase ``internal data set'' is used to maintain the distinction between the SDDS data set and
the representation of data from an SDDS data set within {\tt sddsplot}.  Associated with each
internal data set is the request number, the filename, the file number within the request, the y
dataname, the name group index within the request, the starting page number from the file, and an
optional user-specified tag value (from the commandline or a parameter in the file).  These values
may be used to sort and group the data in order to place on individual panels sets of similar data
from multiple sources.  An instance of this is shown in the last example above.

\item {\bf synopsis:} 
\begin{flushleft}{\tt
sddsplot [{\em X11Switches}] [{\em commonSwitches}]
{\em plotRequestSwitch} {\em fileNames} {\em localSwitches}
[{\em plotRequestSwitch} {\em fileNames} {\em localSwitches} \ldots]
}\end{flushleft}

The {\tt sddsplot} command line is organized into three categories.  First, one may issue any of the standard X11
switches (e.g., -geometry).  Second, one may give a set of switches, indicated by {\em commonSwitches}, that will
apply to all subsequest plot requests.

Third, one gives a series of ``plot requests''.  A plot request starts with one of several switches that give the
names of data elements to be plotted.  It continues with the names of one or more files from which this data is to
be extracted.  In addition, one may include various switches that apply only to current plot request.  These may,
for example, override any common switches that were set prior to the first plot request.  In general, any switch
may be given as a common switch (so that it applies to all plot requests unless overridden) or as a local switch.

In the examples above, only a single plot request is exhibited.  There are no X11 switches and no common switches
set.  The plot request is initiated by the {\tt -columnNames} switch.  The {-graphic} and {-legend} switches are
local switches.

\item {\bf switches:}
\begin{itemize}
\item Initiating a plot request:\\
  \begin{itemize}
  \item {\tt -columnNames={\em xName},{\em yNameList}[,\{{\em y1NameList} | {\em x1Name},{\em y1NameList}\}] }---
        Specifies the names of columns to be plotted.  {\em xName} may be the name of a numeric or string column,
        which is normally plotted against the horizontal or x axis.
        {\em yNameList} gives the comma-separated, optionally wildcarded names of one or more columns of numeric
        data.  Data for each item in {\em yNameList} will be paired with the x data for plotting.

        Some types of plotting require additional data, such as error bars or vector components.  These are
        specified with the {\em x1Name} and {\em y1NameList}.  Each item in {\em y1NameList} is paired with the
        corresponding item in {\em yNameList}; the lists must have the same length.  The interpretation of the
        additional data is specified with the {\tt -graphic=error} or {\tt -arrow} switches. For error bar
        plotting, one may give error bars for both x and y by giving {\em x1Name} and {\em y1NameList}, or for y
        only by giving {\em y1NameList}.  For arrow plotting, giving {\em y1NameList} only is allowable for
        vectors perpendicular to the page. Giving both {\em x1Name} and {\em y1NameList} is required for vectors
        in the plane of the page.

        One may give several {\tt -columnNames} switches in a row in order to specify additional ``datanames'' for
        the request.  This may be convenient if, for example, one wants several different x variables.
  \item {\tt -xExclude={\em xNameList}}---
        specifies the names of x columns to be excluded from the plot. 
        {\em xNameList} gives the comma-separated, optionally wildcarded names of one or more columns.
  \item {\tt -yExclude={\em yNameList}}---
        specifies the names of y columns to be excluded from the plot. 
        {\em yNameList} gives the comma-separated, optionally wildcarded names of one or more columns.
  \item {\tt -toPage={\em pagenumber}}---
        specifies the page number to which sddsplot plots.
  \item {\tt -fromPage={\em pagenumber}}---
        specifies the page number from which sddsplot plots.      
  \item {\tt -parameterNames={\em xName},{\em yNameList}[,\{{\em y1NameList} | {\em x1Name},{\em y1NameList}\}]}---
        Identical to {\tt -columnNames}, except it specifies parameter data to be plotted.  As with {\tt -columnNames},
        several such options may be given in a row in order to add datanames.

  \item {\tt -keep[=\{names | files\}]} ---
        Rarely used.
        Specifies starting a new plot request, but retaining certain information from the previous request.
        If given without qualifiers, the datanames (as specified by {\tt -columnNames} or {\tt -parameterNames})
        and filenames from the previous request are kept; this allows plotting the same data again in a different
        way.  If the {\tt names} qualifier is given, the datanames from the previous request are retained.
        If the {\tt files} qualifier is given, the filenames from the previous request are retained.

  \item {\tt -mpl[=noTitle][,noTopline]} ---
        Provided for compatibility with an older type of data file and rarely used.
        Allows plotting of {\tt mpl} data files with {\tt sddsplot}.  The x and y columns of the {\tt mpl} file
        are used.  The qualifiers may be employed to inhibit use of the {\tt mpl} plot title and topline.

  \item {\tt -namescan=\{all | first\}} --- Specifies whether {\tt sddsplot} should scan all input files when searching
        for matches to wildcard datanames, or only the first.  The default is to scan all files, which may be slow
        for many files with large numbers of columns or parameters.
  \end{itemize}
   
\item Controlling output type:
  \begin{itemize}
  \item {\tt -listDevices}---Lists the names of available graphics devices to the standard error output.
  \item {\tt -device={\em deviceName}[,{\em deviceArguments}]}---Specifies the name of the graphics
        device, plus optional device-specific arguments.  The default device is ``motif'', unless the
        {\tt SDDS\_DEVICE} environment variable if defined, in which case the default device is the one
        named.
        Some commonly-used devices that have device-specific arguments are:
        \begin{itemize} 
        \item {\tt motif} --- The device arguments are a single string of space-separated entries of the
        form {\tt -{\em resourceName} {\em value}}.  These are passed directly to the MOTIF ``outboard-driver''
        without any interpretation.For example,{\tt "-dashes {\em 1}"} qualifier sets the line types with 
        built-in dash styles; {\tt "-linetype {\em linetypeFileName }"} forces MOTIF "outboard-driver" to use
        the user-defined line types (color,dash,thickness)  in a SDDS file intead of the default line types. 
        Other resource names may be found in the help for the driver. 
        \item {\tt postscript},{\tt cpostscript} --- Four qualifiers are presently accepted. {\tt dash} sets
        the cpostscript device to use built-in dash styles for the line drawing. 
        {\tt linetypetable={\em linetypeFileName}} replaces the default line types with the customized line types
        defined in a SDDS file. {\tt onblack} and {\tt onwhite} set the background of the plot.
        \item {\tt png} --- PNG devices accept {\tt rootname}, {\tt template}, {\tt onwhite}, {\tt onblack}, 
        {\tt dash} and {\tt linetypetable}  device arguments. {\tt rootname={\em string}} specifies a rootname
        for automatic filename generation; the resulting filenames are of the form {\em rootname}.DDD, where DDD 
        is a three-digit integer. {\tt template={\em string}} provides a more general facility; one uses it to
        specify an sprintf-style format string to use in creating filenames. For example, the behavior obtained
        using {\tt rootname={\em name}} may be obtained  using {\tt template={\em name}.\%03ld}.
        \item {\tt mif} --- Three qualifiers are presently accepted.  {\tt linesizeDefault={\em size}} sets the
        default line thickness (normally 0.25).  {\tt dashsizeDefault={\em size}} sets the default dash size
        (normally 1.0).  {\tt lineIncrement={\em value}} sets the line thickness increment between different
        line types.
        \item {\tt gif}, {\tt tgif}, {\tt sgif}, {\tt mgif}, {\tt lgif} --- No longer supported, use {\tt png}, 
        {\tt tpng}, {\tt spng}, {\tt mpng}, {\tt lpng} instead.
        \end{itemize}
  \item {\tt -output={\em filename}}---Specifies the name of a file to which graphics output will be sent.
        Used primarily for hardcopy devices (e.g., Postscript) where the data will be sent to a printer.
        By default, the data for such devices is printed to the standard output.
  \end{itemize}
\item Controlling type of plotting:
  \begin{itemize}

  \item {\tt -graphic={\em element}[,type={\em integer}][,fill][,subtype=\{{\em integer} | type\}]}
        {\tt [,connect[=\{{\em linetype} | type | subtype\}]][,vary[=\{type | subtype\}]]}
        {\tt [,scale={\em factor}][,\{eachFile | eachPage | eachRequest | fixForName | ]}
	{\tt fixForFile | fixForRequest\}]} ---
        Specifies the type of graphic element to use for data in the present plot request.  

{\em element} may be one of {\tt line}, {\tt symbol}, {\tt errorBar}, {\tt impulse}, {\tt yimpulse}, {\tt
bar}, {\tt ybar}, {\tt dot}, or {\tt continue}.  These are largely self-explanatory.  {\tt continue}
specifies continuing whatever was done in the previous request.  {\tt impulse} is a line extending from
y=0 to the data value, while {\tt bar} is a line extending from the bottom of the plot region to the data
value.  {\tt yimpulse} and {\tt ybar} are analogous except that the line extends from x=0 or from the
left-hand vertical border of the plot.

The {\tt type} field for the graphic element has different meanings for different elements.  For
lines, impulses, bars, and dots, the type is the color or line style used, depending on the device.
For most devices, values between 0 and 15 inclusive given unique lines.  For symbols and error bars,
the type specifies the style of symbol or error bar to use; the value is between 0 and 8 inclusive
for symbols and between 0 and 1 inclusive for error bars.  For symbols, one may give the {\tt fill}
qualifier to get filled-in symbols.

The {\tt subtype} field is meaningful only for symbols, error bars, and dots.  It specifies the line style
or color to be used in making a symbol or error bar, and the size for a dot.  As for the type field for
line plotting, the value may be between 0 and 15 inclusive.  The {\tt connect} qualifier is also valid for
symbols and error bars only.  It specifies that the symbols and error bars should be connected by lines.
By default, the line type used is 0.

If one desires automatic variation of the line color, symbol type, and
so on, one may obtain this using the \verb|vary| qualifier.  By
default, the type is varied.  The \verb|eachFile|, \verb|eachPage|,
\verb|eachRequest|, \verb|fixForName|, \verb|fixForFile|, and \verb|fixForRequest|
qualifiers may be given to specify how to
assign type or subtype.  For \verb|eachFile|, variation is done
separately for data from different files.  For \verb|eachPage|,
variation is done separately for data from different pages (hence,
items from different pages would have the same line or symbol).  For
\verb|eachRequest|, variation is done separately for each request.
The \verb|fixForName| qualifier in constrast assigns fixed graphic
attributes to items according to the y name.  The \verb|fixForFile|
qualifier assigns fixed graphic attributes for items according to which
data file they are from.  The \verb|fixForRequest| qualifier assigns fixed
graphic attributes according to the request in which the data originates.

  \item {\tt -arrowSettings=[,autoScale][scale={\em factor}][,linetype={\em integer}]}
{\tt [,centered][,singleBarb][,barbLength={\em value}][,barbAngle={\em value}]}
{\tt [,\{cartesianData | polarData | scalarData\}]}
---Specifies parameters for plotting vectors using arrows.

{\tt autoScale} specifies that the scale factor for the length of arrows should be chosen
automatically; if several data pages are being plotted separately, the same scale is used for all of
them.  {\tt scale} may be used instead of {\tt autoScale} to set the factor manually; if both are
given, then the factor given with {\tt scale} multiplies that computed by {\tt autoScale}.

{\tt linetype} specifies the line type to use for the arrows, using the same mechanism as for lines
in the {\tt -graphic} switch.  The default is 0.

{\tt cartesianData}, {\tt polarData}, and {\tt scalarData} specify the type of data being provided.
For the first two, one must have specified both {\em x1Name} and {\em y1NameList} in the plot
request; for {\tt cartesianData}, x1 and y1 are the x and y vector components, while for {\tt
polarData} x1 is the length and y1 is the angle in radians from the positive x direction.

{\tt centered} specifies that arrows should be centered on the corresponding (x, y) point; by
default, the arrow starts at the (x, y) point.  {\tt singleBarb} specifies that arrows should have
only a single barb, rather than the default two barbs; this can be significantly faster for large
amounts of data.  {\tt barbLength} and {\tt barbAngle} specify the length and angle of arrow barbs;
the barb length is a specified as a fraction of the arrow length, which the barb angle is specified
in degrees.

  \item {\tt -linetypeDefault={\em integer}}--- Specifies the default line type for borders, legend
        text, labels, axes, and so on.  If not given, 0 is used.

\end{itemize}

\item Controlling the plotting region:
  \begin{itemize} 

   \item {\tt -scales={\em xmin},{\em xmax},{\em ymin},{\em ymax}}---Specifies the region of the plot in user's
coordinates.  If {\em xmin} and {\em xmax} are equal, then autoscaling is used in x, and similarly for y. Note
that data outside the specified region is still plotted, so that proper clipping of lines occurs.
 
 \item {\tt -range=[\{x|y\}Minimum={\em value}][,\{x|y\}Maximum={\em value}][,\{x|y\}Center={\em value}]}
        --- Constrains the extent
        and center of the plot in user's coordinates.  {\tt xminimum} specifies the minimum allowable
        horizontal extent of the plot; if the autoscaled (or user-specified) range is less than this, the
        range is increased symmetrically to this value.  Similarly, {\tt xmaximum} specifies the maximum
        allowable horizontal extent of the plot.  {\tt xcenter} specifies the center of the horizontal
        range without affecting the extent.  The {\tt y} options are the same, but for the vertical
        coordinates.
  \item {\tt -unsuppressZero[=x][,y]}---Specifies that x=0 and/or y=0 should be within the region of
the plot.  If given without qualifiers, both x and y are ``unsuppressed''.

  \item {\tt -sameScale[=x][,y][,global]}---Specifies that separate panels of data shall be displayed
on the same scales.  In other words, any autoscaling is done based on all of the data from a request,
rather than simply the data on a particular plot panel.  If given without these qualifiers, both x
and y are affected.  {\tt global} forces {\tt sddsplot} to impose the desired condition across all
plot requests.

  \item {\tt -zoom=[\{x|y\}Factor={\em value}][,\{x|p\}Center={\em value}][,\{y|q\}Center={\em
value}]} --- Specifies zoom and pan starting from the scales set by autoscaling or by {\tt -scales}.
A factor less than (greater than) unity zooms out (in).  For each dimension, one may specify the
center of the plot using either the

  \item {\tt -aspectRatio={\em value}} --- Specifies the y/x aspect ratio of the plot.  The value
must be nonzero.  If it is positive, then the desired aspect ratio is obtained by altering the
pspace.  If it is negative, the desired aspect ratio (the absolute value of the value given) is
obtained by altering the data coordinate range.

  \item {\tt -pSpace={\em hMin}{,\em hMax}{,\em vMin}{,\em vMax}}---This option is seldom used, but allows
control of the region of the panel that is mapped to data coordinates, said region being the ``plot space''
or ``pspace''.  The first two coordinates give the horizontal extent, while the second two give the 
vertical extent.  The coordinate values are between 0 and 1.  The defaults are [0.15, 0.9]x[0.15, 0.9].
  \end{itemize}
\item Controlling axes, numeric labels, ticks, and grids:
  \begin{itemize}
  \item {\tt -axes[=x][,y][,linetype={\em integer}]}---Specifies that axes will be placed on the plot,
if they are visible.  By default, both x and y axes are created, with the same linetype as the labels,
scales, and plot border.  One may select a given axis by supply the {\tt x} or {\tt y} qualifier.
One may specify the line type to use for the axes using the {\tt linetype} qualifier.

  \item {\tt -tickSettings=[,[\{x|y\}]grid][[\{x|y\}]spacing={\em value}]} {\tt
[,[\{x|y\}]factor={\em value}][,[\{x|y\}]modulus={\em value}]} {\tt [,[\{x|y\}]size={\em
fraction}][,[\{x|y\}]linetype={\em integer}]} {\tt [,[\{x|y\}]logarithmic]} --- Specifies how to make
ticks and numeric labels for the x and y dimensions. All of the qualifiers have an {\em x} and {\em
y} variant, e.g., {\tt xgrid} and {\tt ygrid}.  Some have a variant that includes both x and y (e.g.,
{\tt grid}).  In the case of the grid option, {\tt xgrid} specifies grid lines rather than ticks for
the x dimension, {\tt ygrid} is similar for the y dimension, and {\tt grid} specifies grid lines in
both dimensions.

The {\tt factor} qualifiers specify factors to apply to the data values in producing the labels.  For example, one
might want to muliply small values by a power of ten in order to get labels that are of order units.  The {\tt
spacing} values give the spacing of the ticks and labels with any factor included.  I.e., to keep the same number
of ticks, {\tt factor} and {\tt spacing} values must be increased together.  Usually, giving the {\tt spacing}
qualifiers is unnecessary, since {\tt sddsplot} chooses appropriate values.

The {\tt modulus} qualifiers allow printing the modulus of the label value rather than the value itself; for
example, one might use {\tt xmodulus=24} if x was the time in hours over many days.  The {\tt size} qualifiers
permit specification of the size of the ticks as a fraction of the range in the opposing dimension; the default is
0.02.  The {\em linetype} qualifiers specify the linetype to be used for ticks and grid lines, using integer values
as for the {\tt -graph=line} switch.  The {\tt logarithmic} qualifiers specify log-style ticks and labels; the
implication is that the data being plotted is the base-ten logarithm of something.

  \item {\tt -subTickSettings=[[\{x|y\}]divisions={\em integer}][,[\{x|y\}]grid]} {\tt [,[\{x|y\}]linetype={\em
integer}][,[\{x|y\}]size={\em fraction}][,xNoLogLabel][,yNoLogLabel]}---Specifies whether and how to make subticks or subgrid lines for the
x and y dimensions.  All of the qualifiers have two or more variants, one that applies to x, one that applies
to y, and (in some cases) one that applies to both.  For example, {\tt xgrid} requests grid lines for x, {\tt
ygrid} requests grid lines for y, and {\tt grid} requests grid lines for both x and y.  The {\tt divisions}
qualifiers specify the number of subdivisions of the major tick intervals; the default is none.  The {\tt linetype}
qualifiers specify the line type to use for subticks or subgrid lines.  The {\tt fraction} qualifiers specify
the size of the subticks as a fraction of the plotting region; the default is 0.01. {\em xNoLogLabel} and {\em yNoLogLabel} specify whether plot subtick labels for log scale axis, they are only valid if the axis uses log scale.

  \item {\tt -yScalesGroup=\{ID={\em string} | fileIndex | nameIndex | nameString | page | request | tag | subpage | iNameString\}} --- 
        Specifies multiple vertical scales.  The most common form is {\tt -yscalesGroup=namestring}, which
        uses a separate scale for every separately-named quantity.  Otherwise, one specifies a separate
        scale for items from different files (by file index), with different name index, different page,
        and so on.  The \verb|tag| is a quality of a dataset specified with the \verb|-tag| option.
        \verb|iNameString| is the name string in inverse order (i.e., so that one compares namestrings
        starting at the end rather than the beginning).  These qualifiers are shared with the
        \verb|groupBy| and \verb|separate| options.
        
  \item {\tt -xScalesGroup} --- Identical to \verb|yScalesGroup| but for x axis scales.

  \item {\tt alignZero[=\{xcenter|xfactor|pPin={\em value}\}][,\{ycenter|yfactor|qPin={\em value}\}]} ---
  This option is provides a facility for lining up zeros on plots with multiple axes.  You must give
at least one of the qualifiers.  The {\tt xfactor} and {\tt yfactor} qualifiers request multiplication
of the upper and lower limits for each scale by the smallest factors that will line up the zeros.
The {\tt xcenter} and {\tt ycenter} qualifiers position the zeros at the center of the plot space,
which may result in empty regions on the plot.  The {\tt pPin} and {\tt qPin} allow specifying the
point at which to ``pin'' the zeros, in plot-space coordinates (0 to 1).

  \item {\tt -grid[=x][,y]}---This option is superseeded by the {\tt -tickSettings} option.  It permits specification
that grids (rather than ticks) will be used for major divisions.

  \item {\tt -noScales}---Specifies that no scales (i.e., no ticks, subticks, or numeric labels) will be plotted.

  \item {\tt -noBorder}---Specifies that no border will be made around the plot region.  Implies {\tt -noScales}.

  \end{itemize}
\item Controlling text labels:
  \begin{itemize}

  \item {\tt -xLabel=[\{@{\em parameterName} | {\em string}\} | use=\{name | symbol |
description\}][,units][,offset={\em value}][,scale={\em value}][,edit={\em string}]}---Controls size,
placement, and content of the x dimension label, which appears directly under the scale labels.  The
default text is of the form {\tt {\em symbol} ({\em units})}, where the symbol and units are taken from
the column or parameter definition fields in the SDDS header for the x data.  If the symbol is blank, then
the element name is used.  Alternatively, the text may be taken from a named string parameter, or from a
string that is given explicitly, or the user may specify with the {\tt use} qualifier that the
element name, symbol, or descrpition be used.  The user may also force the appearance of the units on the
label using the {\tt units} qualifier.   The label text may be edited using Toolkit editing commands
(\progref{SDDS editing}).

The {\tt offset} and {\tt scale} qualifiers allow changing the position and size of the label.  The {\tt
offset} is specified as a fraction of the vertical dimension of the plot region.  The {\tt scale} is
simply a multiplicative factor.

Note that if the value of the parameter {\em parameterName} changes from page to page in a file, and if separate pages are
plotted in different panels, then the label for each panel will be different.  If the pages are plotted together, the value
of the parameter from the first page will be used.

  \item {\tt -yLabel}---This switch has identical usage to {\tt -xLabel}.  {\tt -yLabel} controls the y dimension label.  The
default text contains the y data names of all the columns and parameters being displayed.  If the data all have the same
units, the units are displayed as well.  This information is taken from the appropriate entries in the SDDS header.
The {\tt offset} qualifier gives the label offset as a fraction of the horizontal dimension of the plot region.

  \item {\tt -verticalPrint=\{up | down\}}---Specifies the direction of print for the y dimension label.
The default is {\tt up}.

  \item {\tt -title}---This switch has identical usage to {\tt -xLabel}.  The default text is from the {\tt contents}
field of the {\tt description} command in the first file from which data is displayed.  
  \item {\tt -topTitle}---Normally, the title goes below the x dimension label.  This switch directs that it be placed
at the top of the plot, above the ``topline label''.
  \item {\tt -topline}---This switch has identical usage to {\tt -xLabel}.  It is blank by default.
  \item {\tt -filenamesOnTopline}---Directs that the topline text contain the names of the files from which data
is displayed.
  \item {\tt -labelSize={\em fraction}} --- {\bf Obsolete}: 
Specifies a common size for all labels, including numeric labels.  
In the original version of {\tt sddsplot}, the {\em fraction} was the horizontal size of the characters as a 
fraction of the horizontal size of the plot region.  This meaning is no longer precisely true because the
new version doesn't used fixed character sizes.  However, this option may still be used to scale character
sizes up and down.  The previous nominal value for {\em fraction} was 0.03, which is now used as the reference
point for scaling.  Hence, if you specify 0.06, the character sizes would be doubled.  
  \item {\tt -noLabels}---Specifies that no labels (i.e., x and y dimension labels, title, and topline label) will be
made.  

  \item {\tt -string=\{@{\em parameterName} | {\em string}\},\{x|p\}Coordinate={\em value}}
{\tt \{y|q\}Coordinate={\em value}[,scale={\em factor}][,angle={\em degrees}]}
{\tt [,justifyMode={\em mode}][,linetype={\em integer}][,edit={\em string}]} ---
Specifies display of string data on the plot.  The string may either
be extracted from a named string parameter or given explicitly.  If the value of the parameter {\em parameterName} changes
from page to page in a file, and if separate pages are plotted in different panels, then the label for each panel will be
different.  If the pages are plotted together, the value of the parameter from the first page will be used.

The coordinates of the string may be specified either in users coordinates (i.e., x and y), or unit coordinates (i.e., p and
q); the unit coordinates are (0,0) at the lower left of the plot region and (1,1) at the upper right.  {\tt scale} permits
changing the size of the letters by a specified factor. {\tt angle} permits changing the angle of the string; a value of 90
gives upward vertical print.  

Normally, text is ``left bottom'' justified, which means that the coordinates given are those of the left bottom corner of
the first letter of the string.  Justification may be changed with the {\tt justifyMode} qualifier, which accepts a mode string
of the form {\tt \{ l | r | c\}\{t | b | c\}}.  The letters stand for Left, Right, Center, Top, and Bottom, respectively.
The default justification would thus be specified as {\tt justify=lb}.

The text is normally creating using line type 0.  This may be changed with the {\tt linetype} option.  As with the other
labels, the text may be edited using Toolkit editing commands (\progref{SDDS editing}).

  \item {\tt -dateStamp}---Directs that a time and date stamp be placed on the plot.  It appears in the upper left corner
of the plot.

  \end{itemize}
\item Altering or rearranging data prior to plotting:
  \begin{itemize}

  \item {\tt -swap}---Specifies that the x data will be plotted as y and vice-versa.  

  \item {\tt -transpose}---Specifies that the data matrix be transposed prior to plotting.  This
means, for example, that if the plot request specified N columns of y data and if the table contained
M rows, one would get a plot of M quantities as a function of the index of the column.  The implicit
assumption is that the N columns contain comparable quantities.  This would allow one to display, for
example, how the quantities changed from row to row in the data.  Each row of data thus organized is
marked as a separate ``subpage'' (see the {\tt -groupBy} and {\tt -separate} switches), so that one
can for example split rows onto separate panels.

  \item {\tt -factor=[\{x|y\}Multiplier={\em value}]} --- Specifies that the x
and/or y data for the present request will be multiplied by the given values.  Note that it is the
users responsibility to ensure that the units that are displayed are corrected, if required.

  \item {\tt -offset=[\{x|y\}Change={\em value}][,\{x|y\}Parameter={\em
value}][,\{x|y\}Invert]} --- Specifies that the x and/or the y data be
offset by either specified values, qor by values in named numerical
parameters.  Normally, the offset is of the form $x \rightarrow
x+x_o$.  The {\tt invert} qualifiers cause the offset to be subtracted
rather than added.

If {\tt -factor} is given together with {\tt -offset}, then the offset is applied first.

  \item {\tt -mode=\{x | y\}=\{linear | logarithmic | autolog | normalize | offset | coffset | center | meanCenter | fractionalDeviation | specialScales\}[,...]} ---

Invokes one or more standard transformations of data, independently
for x and y values.  The {\tt linear} mode is the default.  {\tt
normalize} mode directs that data be displayed after independent
normalization to the interval [-1, 1]; to do this, the data is divided
by the maximum absolute value in the data.  The {\tt offset}, {\tt
coffset}, {\tt center}, and {\tt meanCenter} qualifiers result in 
shifting of the data: {\tt offset} directs that data be shifted so 
that the first value plotted is zero; {\tt coffset} directs the data 
to use a common offset from the first plot; {\tt center} directs that 
data be shifted the center of the range is zero; {\tt meanCenter} 
directs that the data be shifted so that the average plotted value 
is zero.

{\tt logarithmic} mode implies that the base-ten logarithmic of the
appropriate values is taken prior to plotting.  Normally, this does
not produce log-type scales; use of the {\tt specialScales} keyword
together with the {\tt logarithmic} keyword will obtain this. One can
also use the {\tt -tickSettings} option for this, which is the preferred
method.  {\tt autolog} mode results in choice of linear or log-scale plotting
based on the range of the data.  If the range of the data is more than a factor
of 15, then log mode is used (with log scales).  Otherwise, linear mode is used.

{\tt fractionalDeviation} plots the data after subtracting and then dividing by the mean value.

\item {\tt -stagger=[xIncrement={\em value}][,yIncrement={\em value}][,files][,datanames]} ---
Directs that data displayed on the same panel will be incrementally offset for display.  This is
useful in order to make mountain range plots, or to offset similar data for clarity.  {\tt
xIncrement} and {\tt yIncrement} are used to specify the increments for each dimension; zero is the
default.  Normally, only data from the same column or parameter is staggered, with the stagger amount
increasing with each page in the file.  The {\tt files} qualifier directs incrementing the offset
when plotting proceeds to a new file on the same panel.  The {\tt datanames} qualifier directs
incrementing the offset when plotting proceeds to a new dataname (i.e., column or parameter name)
within the same file on the same panel.

  \item {\tt -enumeratedScales=[interval={\em integer}][,limit={\em integer}][,scale={\em
factor}]} {\tt [,allTicks][,rotate][,editCommand={\em string}]} --- Allows control of the display of
enumerated value strings when the x data is of string type.  {\tt interval={\em N}} specifies
displaying and making a tick for every $N {th}$ enumerated value; the default is 1.  Also, {\tt
limit={\em M}} specifies displaying and making a tick for only $M$ enumerated values at equal
spacing; the default is unlimited.  If one of these options is employed but one desires to see all
the ticks (even those without labels), the {\tt allTicks} qualifier may be given.  {\tt scale}
specifies a factor by which to increase the size of the text.  {\tt rotate} specifies rotation of the
printed text from the normal orientation to the optional orientation; if enumerated data is displayed
along the x dimension, the normal (optional) orientation is vertical (horizontal) printing.  These
are reversed if the enumerated data is displayed along the y dimension.

  \end{itemize}
\item Creating legends and data labels:
  \begin{itemize}

  \item {\tt -legend[=\{ \{x|y\}Symbol | \{x|y\}Description | \{x|y\}Name | filename | 
 specified={\em string} | parameter={\em name}\}] [,editCommand={\em string}]
  [,firstFileOnly][,scale={\em factor}]} \rm
--- Specifies creation of a legend for the datanames in the current request.  By default, the
legend text is the symbol field for the y data; if the symbol is blank, the dataname is used.  {\tt
xSymbol} and {\tt ySymbol} specify use of the x or y data symbols, or the datanames if the requested
symbol is blank.  {\tt xDescription} and {\tt yDescription} specify use of the indicated description
fields.  {\tt xName} and {\tt yName} specify use of the indicated datanames.  {\tt filename}
specifies use of the name of the file from which the data comes.  {\tt specified={\em string}}
specifies use of the given string.  {\tt parameter={\em name}} specifies use of the contents of the
named string parameter.  Any legend text may be editing using SDDS editing commands\progref{SDDS
Editing} via the {\tt editCommand} qualifier.  If {\tt firstFileOnly} is given, only the first file
in the request will have legends generated.  If {\tt scale={\em factor}} is given, the legend text
size is scaled by the given factor.

  \item {\tt -lSpace={\em qmin},{\em qmax},{\em pmin},{\em pmax}}---Specifies the region in which
legends will be placed.  The coordinates are pspace coordinates.  Since the legends are typically
outside the pspace, the coordinates may be greater than unity.  For example, the default values are
[1.02, 1.18]x[0.0, 1.0].  This option is usually used to place the legend inside the pspace, or to
extend the size of the lspace to accomodate long legend text.

  \item {\tt -pointLabel={\em name}[,edit={\em editCommand}][,scale={\em number}]
        [,justifyMode=\{rcl\}\{bct\}]} --- Specifies labeling of individual data points using
        data from column or parameter {\em name}.  The labels may be edited by specifying an
        {\em editCommand} with the {\tt edit} qualifier.  The {\tt scale} qualifier may be
        used to scale the label size.  The {\tt justifyMode} qualifer is used to change the
        location of the label relative to the point; the first letter gives the horizontal
        justification (right, center, or left) and the second gives the vertical
        justification (bottom, center, or top).
        
  \end{itemize}

\item Creating overlays:

The overlay feature allows displaying data that has different scales
on the same plot.  In most cases, it is superseded by the {\tt
-yScalesGroup} and {\tt -xScalesGroup} options.  The only exception is
when one wants to overlay data without having scales shown for the
data.  (An example is plotting magnet layouts for Twiss parameter
plots using the {\tt magnets} output from {\tt elegant}.)

{\tt -overlay=[\{x|y\}mode={\em mode}][,\{x|y\}factor={\em value}]} {\tt [,\{x|y\}offset={\em
value}]} {\tt [,\{x|y\}center]}---Normally, {\tt sddsplot} displays all data on a single panel on the same
scale.  In some cases, one wants to overlay data that is on a different scale from other data on the
panel.  One way to do this is with the {\tt -overlay} switch, which gives convenient control of how
overlayed data is displayed.  Any data in a plot request for which this switch is given will be
overlayed as specified.

The {\tt xmode} and {\tt ymode} options allow two types of scaling for x and y independently.  A mode
of {\tt normal} means that the indicated data is treated normally.  The default mode is {\tt unit},
which means that the data is scaled so that its full range is equal to the full coordinate range of
the plot in the appropriate (x or y) dimension.

The data is further adjusted according to any additional qualifiers given.  The {\tt center}
qualifiers offset the data so that the data is centered in the plot space; normally, zero in the data
is mapped to zero in the user's coordinates.  The {\tt factor} qualifiers scale the data by the given
factor about the center value.  The {\tt offset} qualifiers offset the data by specified amounts; if
{\tt mode=normal}, the offset is in user's coordinates, otherwise it is in pspace coordinates.

Users needing only the {\tt factor} facility should consider the {\tt -factor} switch, since it is
easier to use.

\item Controlling plot panels:
  \begin{itemize}
  \item {\tt -newPanel}---Specifies that the current plot request will start a new plot panel.
  \item {\tt -endPanel}---Specifies that the current plot request will end the current plot panel.

  \item {\tt -layout={\em hNumber},{\em vNumber}[,limitPerPage={\em integer}]}---Specifies the layout
of panels on each plot page.  The maximum number of panels on any page is the product of {\em
hNumber} and {\em vNumber}, which are the number of panels horizontally and vertically, respectively.
The default is {\em hNumber}=1 and {\em vNumber=1}.  If {\tt limitPerPage} is given, then only the
specified number of panels will appear on any page; for example, {\tt -layout=2,2,limit=3} would
imply three panel spaces per page, with one left blank.

  \end{itemize}

\item Grouping, sorting, and separating data:

  \begin{itemize} \item {\tt -sever[=xGap={\em value}][,yGap={\em value}]}---For line plotting, {\tt
sddsplot} will normally connect points sequentially without regard for gaps in the data.  The {\tt
-sever} switch specifies various means of locating gaps in data and directs lifting the ``pen''
whenever a gap occurs.  If {\tt -sever} is given without qualifiers, the pen is lifted whenever the x
value decreases; this is useful for plotting data where the x value is expected to increase
monotonically for each group of points.

The {\tt xgap} and {\tt ygap} qualifiers invoke a more sophisticated and more generally applicable form of
severing.  For each dimension for which severing is requested, the pen is lifted whenever the absolute
difference of two successive values exceeds a defined limit.  This limit is specified either in absolute
or fractional terms using the {\em value} entry.  If {\em value} is positive, the gap threshold is equal
to {\em value}.  If {\em value} is negative, the gap threshold is {\em -value} times the mean spacing
between successive points; a value of -1.5 has been found to work well for data that is roughly equispaced
with occasional missing points.

  \item {\tt -tagRequest=\{{\em number} | @{\em parameterName}\}}---Specifies that data from the
current requested will be tagged with either the given (generally floating-point) {\em number}, or
with the values from the numeric parameter {\em parameterName}.  Using the {\tt -groupBy} and {\tt
-separate} options permits grouping and sorting of data by tag values.  If a data set has multiple
pages in the file, and if pages are split (see {\tt -split} below), then parameter-tagged data will
have the parameter value from the first page in each group of pages.

  \item {\tt
 -groupBy[=request][,tag][,fileIndex][,nameIndex][,page][,subpage]} {\tt [,fileString][,nameString][,iNameString]} ---
 Specifies how internal data sets will be ordered.  {\tt -sortBy} might have been a more appropriate
 name for this switch.  The qualifiers that appear in the list are shown in the order that
 corresponds to the default sorting.  The file index is the sequential number within the request of
 the file from which the internal data set is taken; the file string is the name of the file.  The
 name index is the sequential index within the request of the dataname group for the internal data
 set, while the name string is the name of the y data.  The page is the sequential number in the file
 of the first SDDS data page from which data appears in the internal data set.  The subpage is a
 sequential number within each internal data set, which allows subdivision of the internal data set.
 The request is the sequential number of the plot request that resulted in generation of the internal
 data set.  The tag is a single user-supplied value or a value read from a parameter that is
 associated with each internal data set; by default, all data sets are tagged with the value 0.  If a
 file is split into several internal data sets, each may have a different tag value if the tag is
 read from a parameter; in this case, the data sets are eached tagged with the value for the first
 included data page.

The order in which the qualifiers to {\tt -groupBy} are given determines the priority of sorting by
the various criteria.  In the default ordering, data sets are sorted by request number, subsorted by
tag (usually a null operation unless data is tagged by the user), subsubsorted by file index,
subsubsubsorted by dataname index, etc.  Each successive qualifier results in moving the indicated
sort criterion to the next highest priority.  Any qualifiers not given are retained in the default
order.

If one wanted to bring together, for example, internal data sets with the same data name, one would
give {\tt -groupby=nameString}.  In this case, the new sorting priority would be {\tt nameString},
{\tt request}, {\tt tag}, etc.

  \item {\tt -separate[=\{{\em numberToGroup} | groupsOf={\em number} | fileIndex | fileString |
nameIndex | nameString | page | subpage | request | tag | iNameString\}]} --- Specifies how to separate internal data
sets onto panels.  If given with no qualifiers, each internal data set is placed on a separate panel.
If given with a single integer argument, or with the {\tt groupsOf} qualifier, then the specified
number of data sets appear on each panel; the data sets are assign to panels in the order determined
by {\tt -groupBy} or the default thereof.

If one of the other qualifiers is given, then panel separation occurs when the indicated criterion
changes as the data sets are accessed in sorted order.  Most commonly, one uses {\tt -groupby={\em
criterion}} {\tt -separate={\em criterion}}.  For example, one might want to group by filename and
separate by filename.
 
  \item {\tt -split=\{pages[,interval={\em integer}] | parameterChange={\em name}[,width={\em
value}][,offset={\em value}] | columnBin={\em name},width={\em name}[,start={\em
value}][,completely]\}}---As discussed in the introductory sections, when {\tt sddsplot} reads data
for one dataname group from a file, it normally concatenates data from successive pages to form a
single internal data set.  This would mean, for example, that all of the data from the file would be
displayed with the same linetype or symbol.  The {\tt -split} switch overrides this behavior,
splitting the data into multiple internal data sets.

The simplest and most commonly-used way of doing this is to split the data page boundaries; this is
done using the {\tt -split=pages} mode.  The optional {\tt interval} specifies spliting after a
specified number of page boundaries.  Splitting data does not imply that the data will appear on
separate plot panels, but allows this and other possibilities.  (To separate page-split data onto
panels, one uses {\tt -separate=pages}, as discussed above.)

One can also page-split based on the value of a parameter, using {\tt -split=parameterChange}.  This
directs that a new internal data set will be started wheneven the named parameter changes.  For
numeric parameters, the {\tt width} and {\tt start} qualifiers may be used.  If {\tt width} is
specified, the change must exceed the given value before a split occurs.  If {\tt start} is
specified, the reference value for changes is set to the given value; otherwise, the first parameter
value is used.  (For example, one might wish to split when a parameter changed by 5 units referenced
from 2.5 units, giving boundaries of 7.5, 12.5, etc.; this would be obtained with {\tt
width=5,start=2.5}.)

The {\tt columnBin} mode is different from the other two modes.  Rather than splitting data into internal
data sets at page boundaries, it groups or bins data into subpages according to the value in a specified
numeric column.  (It is appropriate only for plotting column data.)  {\tt columnBin} mode may be used with
{\tt pages} mode to split and subsplit data into pages and subpages.  For example, one might have a data
file with many pages of time-series data.  One might want to plot each page separately, but within
each page one might want to color-code the points according to some value in the table (e.g., a valid-data
indicator). This would be accomplished using {\tt -split=pages,columnBin={\em name},width={\em value}} {\tt
-separate=pages} {\tt -graph=dot,vary,eachPage}.

  \item {\tt -omniPresent}---Specifies that the data sets from the current request will appear on all
plot panels.

  \item {\tt -replicate=\{number={\em integer} | match=\{names | pages | requests| files\}\}\{,scroll\}} --- Specifies replication of a dataset so that it can be plotted several times. This is similar to -{\tt omniPresent}, but more flexible.  When a dataset is replicated,
 each replicant appears to have come from a different page of the original file.  The number of replications is controlled by
 the first option: a specific number of replications can be requested, or it can be asked to replicate a number of times 
 equal to the maximum number of pages in any file, data names in any request, plot requests, or files in any request.  
 If the {\tt scroll} qualifier is given, then the replicants do not have the same number of data points.  Instead, successive
 copies are more and more complete until the final replicant has the full dataset.

  \end{itemize}
\item Winnowing data:
  \begin{itemize}
  \item {\tt -limit=[\{x|y\}Minimum={\em value}][,\{x|y\}Maximum={\em value}][,autoscaling]}---
Specifies limits to be placed on x and y values prior to plotting.  Points beyond the indicated limits are 
eliminated from the data prior to plotting.  This complements the facility available from {\tt -filter} and
{\tt -match} in that one need not specify the name of the data one is winnowing with.  This permits easier
filtering of data from many columns or parameters.

The {\tt autoscaling} qualifier specifies that {\tt sddsplot} will not remove data outside the
defined limits, but rather that it will ignore it for purposes of autoscaling.  If lines are used to
connect data points, this could result in lines being drawn to the boundary of the plot region, thus
showing the presence of extreme points.

  \item {\tt -sparse={\em interval}[,{\em offset}]}---Specifies that only every {\em interval}${
{th}}$ point will be used.  If {\em offset} is not given, the first point in the internal data set is
the first taken; otherwise, the {\em offset}${ {th}}$ point is the first taken.

  \item {\tt -sample={\em fraction}}---Specifies random sampling of data to retain only the indicated
fraction of the points.  {\em fraction} gives the probability that any point will be used.  Hence,
the data actually used may vary from run to run since the random number generator is seeded with the
system clock.

  \item {\tt -clip={\em head},{\em tail}[,invert]}---Specifies removal of {\em head} points from the
beginning and {\em tail} points from the end of each internal data set.  If {\tt invert} is given,
the points that would have been removed are instead the only ones used.

  \item {\tt -presparse={\em interval}[,{\em offset}]}---Similar to {\tt -sparse}, except that
sparsing is done at the time the data page is read and only once for all requests and
datanames that draw data from the data page.  This is faster, and is usually what is desired.
However, if one wants to plot sparsed and unsparsed data from the same file at the same time,
{\tt -presparse} cannot be used.  If both {\tt presparse} and {\tt sparse} are given, both
are applied.

   \item {\tt -filter=\{column | parameter\},{\em rangeSpec}[,{\em rangeSpec},{\em logicOp}...] }
--- Specifies winnowing each internal data set based on numerical data in parameters or columns.  A
{\em range-spec} is of the form {\tt {\em name},{\em lower-value},{\em upper-value}[,!] }, where
\verb|!| signifies logical negation.  A point passes a {\tt column}-based filter if the value in the
named column is inside (or outside, if negation is given) the specified range, where the endpoints
are considered inside.  {\tt parameter}-based filters are similar, except that the point passes only
if the value of the named parameter for the page from which it comes is acceptable.  One or more
range specifications may be combined to give a accept/reject status by employing the {\em
logic-operations}, \verb|&| (logical and) and \verb&|& (logical or).
     \item {\tt -timeFilter=\{column | parameter\},[before=YYYY/MM/DD@HH:MM:SS] [,after=YYYY/MM/DD@HH:MM:SS][,invert]} 
--- Specifies date range in YYYY/MM/DD@HH:MM:SS format in time parameters or columns. The invert option cause the
filter to be inverted, so that the data that would otherwise be kept is removed and vice-versa. For example,
if one want to keep data between 8:30AM on Januaray 2, 2003 and 9:20PM on February 6,2003, the option woould be
     -timeFilter=column,Time,before=2003/2/6@21:20,after=2003/1/2@8:30
assume that the time data is in the column Time.
   \item {\tt -match=\{column | parameter\},{\em matchTest}[,{\em matchTest},{\em logicOp}]} ---
Specifies winnowing based on data in string parameters or columns.  A {\em matchTest} is of the form
{\tt {\em name}={\em matchingString}[,!]}, where the matching string may include wildcards.
If the first character of {\em matchingString} is '@', then the remainder of the string is taken to
be the name of a parameter or column.  In this case, the match is performed to the data in the named
entity.

The use of several match tests and logic is done just as for {\tt -filter}.  For example, to match
all the rows for which the column {\tt Name} starts with 'A' or 'B', one could use
{\tt -match=column,Name=A*,Name=B*,|}.  (This could also be done with {\tt -match=column,Name=[AB]*}.)
  \end{itemize}
\item {Miscellaneous:}
\begin{itemize}
        \item {\tt -repeat[=checkSeconds={\em number}][,timeOut={\em seconds}]} ---
Specifies repeated plotting of data from the files, with replotting occuring when any
file is modified.  By default, {\tt sddsplot} checks the files every second and times out
after 900s of no change.  This is available on UNIX systems only.  It is best used with
the \verb|motif| device type and the following device argument:
\verb|-device=motif,''-movie true -keep 1''|.
        \item {\tt -drawLine=}
        {\tt \{x0Value={\em value}|p0Value={\em value}|x0parameter={\em name}|p0parameter={\em name}\}}
        {\tt \{x1Value={\em value}|p1Value={\em value}|x1parameter={\em name}|p1parameter={\em name}\}}
        {\tt \{y0Value={\em value}|q0Value={\em value}|y0parameter={\em name}|q0parameter={\em name}\}}
        {\tt \{y1Value={\em value}|q1Value={\em value}|y1parameter={\em name}|q1parameter={\em name}\}} ---
        Specifies drawing of lines on the plot by giving the two endpoints of the line.
        For each endpoint (labeled '0' and '1'), one must specify the x or p coordinate (for horizontal) and
        the y or q coordinate (for vertical).  Each coordinate name be specified explicitly (e.g.,
        \verb|x0Value=1.7|) or via a parameter (e.g., \verb|x0parameter=alpha|).  If a parameter is
        given, the coordinate can change as the parameter value changes in the file.
        
\end{itemize}

\end{itemize}

\item {\bf special characters:}
{\tt sddsplot} supports Greek and mathematical characters in labels and strings through special sequences embedded in
text strings.  A similar mechanism is used to allow character-by-character control over size and positioning.
The special sequences are of the form \verb|$|{\em character}, where {\em character} may be one of the following:
\begin{itemize}

\item {\tt a}, {\tt b}, {\tt n}: provide subscript and superscript control.  {\tt a} puts the character Above the
normal position (superscript), {\tt b} puts the character Below the normal position (subscript), and {\tt n}
returns to Normal.

\item {\tt g}, {\tt r}: provide for switching between Greek and Roman
character sets. \verb|$g| switches into Greek mode, while \verb|$r|
switches back to Roman mode.  The correspondance between Greek
characters and the alphabet is shown in Figure \ref{CharSet}.  For
example, to make a lower-case alpha, one would use \verb|$ga$r|.

\item {\tt s}, {\tt e}: provide for switching between Special and
normal characters.  \verb|$s| switches to special character mode,
which provides mathematical and other symbols.  Figure \ref{CharSet}
shows the correspondance between special characters and keyboard
characters.  For example, to make a ${\rm \pm}$ symbol, one would
employ \verb|$sa$e|, while a right-pointing arrow would be obtained
with \verb|$s5$e|.

\newcommand{\PSFigure}[4]{\begin{figure}[htb]
  \vspace{-0.38in}
  \includegraphics[width=\columnwidth,height=#1]{#4}
  \vspace{-0.57in}
  \caption[#2]{#2}\label{#3}
  \end{figure}}

\PSFigure{15cm}{Special character set}{CharSet}{charSet.eps}

\item {\tt i}, {\tt d}: provide for Increasing and Decreasing the
character size.  The two sequences \verb|$i| and \verb|$d| are
inverses of each other.  \verb|$i| increases the size of subsequent
characters by 50\%, while \verb|$d| decreases the size of subsequent
characters by ${\rm 33 \frac{1}{3}}$\%.  These are seldom used, since
{\tt sddsplot} provides other means of controlling the size of
characters in labels and strings.

\item {\tt u}, {\tt v}: provide for motion of the baseline Up and down by one half character height.  

\item {\tt t}, {\tt f}: provide for making Taller and Fatter characters.  \verb|$t| makes characters twice as
tall while maintaining width, while \verb|$f| makes characters half as tall while maintaining width.  

\item {\tt h}: specifies moving back one half space.
\end{itemize}


\item {\bf environment variables:}
        \begin{itemize}
        \item {\tt SDDS\_DEVICE} --- Gives the name of the device type to use as the default.
        \end{itemize}
\item {\bf see also:}
    \begin{itemize}
    \item \hyperref{Data for Examples}{Data for Examples (see }{)}{exampleData}
    \item \progref{SDDS editing}
    \item \progref{SDDS Wildcard Conventions}
    \end{itemize}
\item {\bf author:} M. Borland, H. Shang and R. Soliday ANL/APS.

\item {\bf acknowledgements}: {\tt sddsplot} uses device driver code from the program {\tt GNUPLOT}, 
with modifications and enhancements made at Argonne.  The GNUPLOT code is covered by a separate
copyright, and is used by permission of the authors.  See the {\tt GNUPLOT\_README} file included
with the distribution for restrictions associated with this code.

The GUI X-windows program ({\tt mpl\_motif}) was written by K. Evans of ANL/APS.

The GIF drivers use the {\tt gd 1.2} library by Thomas Boutell.  The latter is copyrighted by the
Quest Protein Database Center, Cold Spring Harbor Labs.

\end{itemize}


