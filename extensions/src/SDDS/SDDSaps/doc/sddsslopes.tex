% $Log: not supported by cvs2svn $
% Revision 1.5  1996/04/28  23:05:52  emery
% Spelling mistake in an switch explanation.
%
% Revision 1.4  1996/04/28  22:58:13  emery
% Corrected the font of the option or file specification.
%
%
% Template for making SDDS Toolkit manual entries.
%
\begin{latexonly}
\newpage
\end{latexonly}

%
% Substitute the program name for <programName>
%
\subsection{sddsslopes}
\label{sddsslopes}

\begin{itemize}
\item {\bf description:}
%
% Insert text of description (typicall a paragraph) here.
%
\verb+sddsslopes+ makes straight line fits of column data
of the input file with respect to a selected column used as independent variable.
The output file contains a one-row data set of slopes and intercepts
for each data set of the input file.
Errors on the slope and intercept may be
calculated as an option.

\item {\bf examples:} 
%
% Insert text of examples in this section.  Examples should be simple and
% should be preceeded by a brief description.  Wrap the commands for each
% example in the following construct:
% 
%
The file corrector.sdds below contains beam position monitors (bpms) readbacks as a 
function of corrector
setting. The defined columns are {\tt CorrectorSetpoint} and the series
{\tt bpm1}, {\tt bpm2}, etc.
The bpm response to the corrector setpoints are calculated with the use of \verb+sddsslopes+:
\begin{flushleft}{\tt
sddsslopes corrector.sdds corrector.slopes -independentVariable=CorrectorSetpoint
   -columns='bpm*'
}\end{flushleft}
where all columns that match with the wildcard expression \verb+bpm*+ is selected
for fitting.
\item {\bf synopsis:} 
%
% Insert usage message here:
%
\begin{flushleft}{\tt
sddsslopes [-pipe=[input][,output]] {\em inputFile} {\em outputFile}
      -independentVariable={\em parameterName} [-range={\em lower},{\em upper}]
      [-columns={\em listOfNames}] [-excludeColumns={\em listOfNames}] 
      [-sigma[=generate]] [-residual={\em file}] [-ascii] [-verbose]
}\end{flushleft}
\item {\bf files:}
% Describe the files that are used and produced
The input file contains the tabular data for fitting. Multiple
data sets are processed one at a time. 
For optional error processing, additional columns of sigma values
associated with the data to be fitted must be present. These sigma column 
must be named {\tt {\em name}Sigma} or {\tt Sigma{\em name}},
the former one being searched first.

The output file contains a one-row data set for each data set in the 
input file. The columns defined have names
such as {\tt {\em name}Slope}, and {\tt {\em name}Intercept} where {\tt {\em name}} is the name of
the fitted data.  If only one file is specified, then the input file is 
overwritten by the output.
A string column called \verb+IndenpendentVariable+ is defined containing the name of the indepedent variable.

\item {\bf switches:}
%
% Describe the switches that are available
%
    \begin{itemize}
    \item {\tt  -pipe[=input][,output]} --- The standard SDDS Toolkit pipe option.
    \item {\tt  -independentVariable={\em parametername} }
        --- name of independent variable (default is the first valid column).
    \item {\tt  -range={\em lower},{\em upper}} --- The range of the independent
        variable where the fit is calculated. By default, all data points are used.
    \item {\tt  -columns={\em listOfNames}}   
        ---  columns to be individually paired with independentVariable 
        for straight line fitting.
    \item {\tt  -excludeColumns={\em listOfNames}}  ---  columns to exclude from fitting.
    \item {\tt  -sigma[=generate]}  
        ---   calculates errors by interpreting column names 
        {\tt {\em name}Sigma} or {\tt Sigma{\em name}} as
        sigma of column {\tt {\em name}}. If these columns don't exist
        then the program generates a common sigma from the residual of a first fit,
        and refits with these sigmas. If option {\tt -sigma=generate} is given,
        then sigmas are generated from the residual of a first fit for all columns,
        irrespective of the presence of columns {\tt {\em name}Sigma} or {\tt Sigma{\em name}}.
    \item {\tt -residual={\em file}} --- Specifies an output file into which the
        residual of the fits are written. The column names in the residual file
         are the same as they appear in the input file.
    \item {\tt  -ascii }    ---  make output file in ascii mode (binary is the default).
    \item {\tt  -verbose }  ---  prints some output to stderr.

    \end{itemize}
%\item {\bf see also:}
%    \begin{itemize}
%
% Insert references to other programs by duplicating this line and 
% replacing <prog> with the program to be referenced:
%
%    \item \progref{<prog>}
%    \end{itemize}
%
% Insert your name and affiliation after the '}'
%
\item {\bf author: L. Emery } ANL
\end{itemize}

