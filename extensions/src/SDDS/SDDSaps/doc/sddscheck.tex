\begin{latexonly}
\newpage
\end{latexonly}
\subsection{sddscheck}
\label{sddscheck}

\begin{itemize}
\item {\bf description:} {\tt sddscheck} is a simple tool to allow checking a file to see if
it is a valid SDDS file, or if it is corrupted.  The primary use is in shell scripts that need
to detect such conditions.  {\tt sddscheck} issues one of four messages: {\tt ok}, {\tt
nonexistent}, {\tt badHeader}, or {\tt corrupted}.  (See \progref{sddsconvert} about recovering
corrupted files.)

\item {\bf examples:} 
Under UNIX, one could do the following to check a file before plotting it:
\begin{flushleft}{\tt
if (`sddscheck APS.twi` == "ok") plotTwissParameters APS.twi
}\end{flushleft}
where {\tt plotTwissParameters} is a hypothetical plotting script.
\item {\bf synopsis:} 
\begin{flushleft}{\tt
sddscheck {\em filename}
}\end{flushleft}
\item {\bf files:}
{\em filename} is the name of a single file to be checked.
\item {\bf switches:}
\begin{itemize}
\item {\tt -printErrors} --- Causes the SDDS error traceback to be printed if the file is not \verb|ok|.
        This may be helpful in determining the problem with the file.
\end{itemize}
\item {\bf see also:}
    \begin{itemize}
    \item progref{sddsconvert}
    \end{itemize}
\item {\bf author:} M. Borland, ANL/APS.
\end{itemize}


