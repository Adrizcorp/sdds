\begin{latexonly}
\newpage
\end{latexonly}
\subsection{sddsgenericfit}
\label{sddsgenericfit}

\begin{itemize}
\item {\bf description:}
{\tt sddsgenericfit} does fits to an arbitrary functional form specified by the user.

\item {\bf examples:} 
Fit a gaussian to a beam profile to get the rms beam size.  In this example, the
file \verb|beamProfile.sdds| is assumed to contain data to be fit in columns
\verb|x| and \verb|Intensity|.
\begin{flushleft}{\tt
sddsgenericfit beamProfile.sdds beamProfile.gfit -column=x,Intensity \\\
``-equation=height x center - sqr 2 / sigma sqr / exp / baseline + Intensity - sqr'' \\\
-variable=name=height,start=1,lower=0,upper=10,step=0.1 \\\
-variable=name=center,start=0,lower=-10,upper=10,step=0.1 \\\
-variable=name=sigma,start=1,lower=0.1,upper=10,step=0.1 \\\
-variable=name=baseline,start=0,lower=-1,upper=1,step=0.1 
}\end{flushleft}
\item {\bf synopsis:} 
\begin{flushleft}{\tt
sddsgenericfit 
[-pipe=[input][,output]] [{\em inputfile}] [{\em outputfile}] 
-columns={\em x-name},{\em y-name}[,ySigma={\em sy-name}] 
-equation={\em string}[,algebraic] [-target={\em value}] [-tolerance={\em value}]
[-simplex=[restarts={\em integer}][,cycles={\em integer},][evaluations={\em integer}]]
-variable=name={\em name},lowerLimit={\em value},upperLimit={\em value},stepsize={\em value},startingValue={\em value}[,units={\em string}][,heat={\em value}]
[-variable=...] [-verbosity={\em integer}] [-startFromPrevious]
}\end{flushleft}

\item {\bf files:} {\em inputFile} contains the columns of data to be
fit.  If {\em inputFile} contains multiple pages, each page of data is
fit separately.  {\em outputFile} has columns containing the
independent variable data and the corresponding values of the fit.
The name of the column is constructed by appending the string {\tt
Fit} to the name of the dependent variable.  The name of the residual
column is constructed by appending the string {\tt Residual} to the
name of the dependent variable.  {\em outputFile} also contains
parameters giving the values of the fit parameters.

\item {\bf switches:}
    \begin{itemize}
    \item {\tt -pipe=[input][,output]} --- The standard SDDS Toolkit pipe option.
    \item {\tt -columns={\em x-name},{\em y-name}[,ySigma={\em name}]} 
        --- Specifies the names of the independent and dependent columns of data, and
        optionally the name of the column containing the errors in the dependent column.
    \item {\tt -equation={\em string}[,algebraic]} --- Specifies the penalty function for
        performing the fit.  If you are fitting the form $y = f(x)$, then the penalty
        function has the form $(y-f(x))^2$.
    \item {\tt -tolerance={\em value}} --- Specifies the minimum change in the (weighted) 
           rms residual that is considered significant
           enough to justify continuing optimization.
    \item {\tt -simplex=[restarts={\em nRestarts}][,cycles={\em nCycles},][evaluations={\em nEvals}]} ---
           Specifies parameters of the simplex optimization used to perform the fit.
           Each start or restart allows {\em nCycles} cycles
           with up to {\em nEvals} evaluations of the function.  
           Defaults are 10 restarts, 10 cycles, and 5000 evaluations.
    \item {\tt -variablename={\em name},lowerLimit={\em value},upperLimit={\em value},stepsize={\em value},startingValue={\em value}[,units={\em string}][,heat={\em value}]} --- 
        Specifies a parameter of the fitting fnuction.  If the \verb|heat| qualifier is
        given, then prior to each restart the code ``heats'' the values by adding random
        numbers to the result of the last iteration.  This can help avoid getting stuck
        in a local minimum.  You must give one \verb|variable| option for every parameter
        of the fit.
    \item {\tt -startFromPrevious} --- Meaningful for multipage input files only.  If given,
        then the optimization for each page starts from the parameter values from the fit
        to the previous page.  By default, fitting for each page starts from the values
        specified on the commandline.
    \item {\tt -verbosity={\em integer}} --- Specifies that informational printouts are desired during fitting.  A larger integer produces more output.
    \end{itemize}
\item {\bf see also:}
    \begin{itemize}
    \item \progref{sddsexpfit}
    \item \progref{sddsgfit}
    \item \progref{sddsoutlier}
    \item \progref{sddspfit}
    \end{itemize}
\item {\bf author:} M. Borland, ANL/APS.
\end{itemize}



