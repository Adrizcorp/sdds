\begin{latexonly}
\newpage
\end{latexonly}
\subsection{sddssequence}
\label{sddssequence}

\begin{itemize}
\item {\bf description:} 
{\tt sddssequence} generates an SDDS file with a single page and several columns of data
of arithmetic sequences. An example application is 
generating values for an independent variable, whose values can be used by 
{\tt sddsprocess} to produce a mathematical function.
\item {\bf examples:} 
Make a file example.sdds with column Index of type long with 100 rows. The value of Index in the first row
is 1 and is incremented by 1 for each successive row.
\begin{flushleft}{\tt
sddssequence example.sdds -define=Index -sequence=begin=1,number=100,delta=1
}\end{flushleft}
\item {\bf synopsis:} 
\begin{flushleft}{\tt
sddssequence [-pipe=[output]] [<outputfile>]
-define=<columnName>[,<definitionEntries>] [-repeat=<number>]
-sequence=begin=<value>[,number=<integer>][,end=<value>]
    [,delta=<value>][,interval=<integer>]
[-sequence=[begin=<value>][,number=<integer>][,end=<value>]
    [,delta=<value>][,interval=<integer> ...]
}\end{flushleft}

\item {\bf files:}
{\em outputfile} is the name of the SDDS file containing data generated.

\item {\bf switches:}
    \begin{itemize}
    \item {\tt -pipe=[output]} --- The standard SDDS Toolkit pipe option.
    \item {\tt -define=<columnName>[,<definitionEntries>]} --- Defines a new column.
        One or more {\tt -sequence} options should follow. 
        Definition entries have the form {\tt fieldName=value}
        where {\tt fieldName} is the name of any namelist
        command field (except the name field) for a column. The default data type is double.
        To get a different type, use {\tt type=<typeName>}.

        Multiple {\tt -define} options can be used to create mutliple columns, each with their
        own set of {\tt -sequence} options.
    \item {\tt -sequence=begin=<value>[,number=<integer>][,end=<value>]}
        {\tt[,delta=<value>][,interval=<integer>]}
        --- Defines the arithmetic sequence for the data column. 
        More that one {\tt sequence} option can be given for a 
        column definition, thus allowing arithmetic sequences of different character in one column.
        The {\tt begin} value must be given in the first {\tt sequence} option.
        If any {\tt sequence} options follows immediately, a default value
        equal to the previous {\tt end} value plus the previous {\tt delta} value is used.
        For the rest of the suboptions, the user must supply 
        ({\tt end}, {\tt delta}), ({\tt end}, {\tt number}), or ({\tt delta}, {\tt number}).
        If {\tt number} isn't supplied, then the set of {\tt start}, {\tt end}, {\tt delta}
        must imply a positive number of rows.

        The {\tt interval} field specifies the number of rows for which the value is frozen
        within the sequence. For instance, 
\begin{verbatim}
sddssequence example.sdds -def=Index,type=long -sequ=beg=1,num=10,end=10,inte=2
sddsprintout -col=Index example.sdds
\end{verbatim}
        produces
\begin{verbatim}
Printout for SDDS file example.sdds
 
  Index    
------------
          1 
          1 
          3 
          3 
          5 
          5 
          7 
          7 
          9 
          9 
\end{verbatim}
    \item {\tt -repeat=<number>} --- Repeats the sequence identically for the given number of times.
    \end{itemize}
\item {\bf see also:}
\item {\bf author:} M. Borland, ANL/APS.
\end{itemize}

