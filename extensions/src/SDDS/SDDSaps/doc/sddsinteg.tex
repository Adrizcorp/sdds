\begin{latexonly}
\newpage
\end{latexonly}
\subsection{sddsinteg}
\label{sddsinteg}

\begin{itemize}
\item {\bf description:}
{\tt sddsinteg} integrates one or more columns of data as a function of a common
column.  The program will perform error propagation if error bars are provided in 
the data set.
\item {\bf examples:} 
Find the integral ${\rm \int \eta_x ds}$ for APS lattices
\begin{flushleft}{\tt
sddsinteg APS.twi APS.integ -integrate=etax -versus=s
}\end{flushleft}
\item {\bf synopsis:} 
\begin{flushleft}{\tt
sddsinteg [-pipe=[input][,output]] [{\em input}] [{\em output}]
-integrate={\em columnName}[,{\em sigmaName}] ...
-versus={\em columnName}[,{\em sigmaName}] [-mainTemplates={\em item}={\em string}[,...]] 
[-errorTemplates={\em item}={\em string}[,...]] 
[-method={\em methodName}] [-printFinal[=bare][,stdout]]
}\end{flushleft}
\item {\bf files:}
{\em input} is an SDDS file containing columns of data to be
integrated.  If it contains multiple data pages, each is treated
separately. The independent quantity along with the requested integrals
is placed in columns in {\em output}.  By default, the
integral column name is constructed by appending ``Integ'' to the
variable column name.  If applicable, the column name for the
integral error is constructed by appending ``IntegSigma''.  

\item {\bf switches:}
    \begin{itemize}
    \item {\tt -pipe[=input][,output]} --- The standard SDDS Toolkit pipe option.
    \item {\tt -integrate={\em columnName}[,{\em sigmaName}]} --- Specifies the name
        of a column to integrate, and optionally the name of the column containing
        the error in the integrand.  May be given any number of times.
    \item {\tt -versus={\em columnName}[,{\em sigmaName}]} --- Specifies the name
        of the independent variable column, and optinally the name of the column
        containing its error.
    \item {\tt -mainTemplates={\em item}={\em string}[,...]} --- Specifies template
        strings for  names and definition entries for the integral columns
         in the output file.  {\em item} may be one of {\tt name}, {\tt description},
        {\tt symbol}.  The symbols ``\%x'' and ``\%y'' are used to represent
        the independent variable name and the name of the integrand, respectively.
    \item {\tt -errorTemplates={\em item}={\em string}[,...]} --- Specifies template
        strings for  names and definition entries for the integral error columns
        in the output file.  {\em item} may be one of {\tt name}, {\tt description},
        the independent variable name and the name of the integrand, respectively.
    \item {\tt -method={\em methodName}} --- Specifies the integration method.  The default method
        is ``trapazoid,'' for trapizoid rule integration.
        Another method is ``GillMiller,'' which is based on cubic interpolation and is much more
        accurate than trapazoid rule; unlike most higher-order formulae, it is not restricted to
        equispaced points.  (See P.E. Gill and G. F. Miller, The Computer Journal, Vol. 15, N. 1, 80-83, 1972.)
        Error propagation is available for trapazoid rule integration only.
        If higher-order integration is needed, one can first interpolate the data with reduced
        spacing of the independent variable using {\tt sddsinterp} with the {\tt -equispaced} option and
        {\tt -order=2} or higher, then integrate using {\tt sddsinteg}.
        This is mathematically equivalent to using a higher-order formula, but is 
        more general as it is not restricted to initially equispaced data.
        However, using Gill-Miller is probably the best approach.
    \item {\tt -printFinal[=bare][,stdout]} --- Specifies that the final value of each
        integral should be printed out.  By default, the printout goes to stderr and
        includes the name of the integral.  If {\tt bare} is given, the names are omitted.
        If {\tt stdout} is given, the printout goes to stdout.
    \end{itemize}
\item {\bf see also:}
    \begin{itemize}
    \item \progref{sddsderiv}
    \item \progref{sddsinterp}
    \end{itemize}
\item {\bf author:} M. Borland, ANL/APS.
\end{itemize}

