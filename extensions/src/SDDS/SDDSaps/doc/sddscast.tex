\begin{latexonly}
\newpage
\end{latexonly}
\subsection{sddscast}
\label{sddscast}

\begin{itemize}
\item {\bf description:} \verb|sddscast| can be ussed to change the data type of the
elements in an SDDS file. 
\item {\bf example:}
Convert the double columns of {\tt APS.twi} to float (note that col1, col2,col3
are the column names in APS.twi file):
\begin{flushleft}{\tt
sddscast APS.twi -cast=col,*,double,float
}\end{flushleft}
\begin{flushleft}{\tt
sddscast APS.twi -cast=col,'(col1,col2,col3)',double,float
}\end{flushleft}
\begin{flushleft}{\tt
sddscast APS.twi -cast=col,'(col1,col4,col5)','(double,float,float)',long
}\end{flushleft}
\begin{flushleft}{\tt
or:
}\end{flushleft}
\begin{flushleft}{\tt
sddscast APS.twi '-cast=col,(col1,col4,col5),(double,float,float),long'
}\end{flushleft}

\item {\bf files:}
{\em inputFile} is an SDDS file containing data to be processed.  The {\em outputFile} argument is
optional.  If it is not given, and if an output pipe is not selected, then the input file will be
replaced.
\item {\bf switches:}
    \begin{itemize}
    \item {\tt -cast={column|parameter|array},<names>,<typeNames>,<newType>}
          names is of the form 'name' (with optional wildcards) or
                     '(name,name,....)'; 
          typeNames is of the form '(long,short,double,float,*)' ;
          newType is long, short,double, or float.
    \item {\tt -pipe=[input][,output]} --pipe flages.
    \item {\tt -noWarnings} --- Suppresses warning messages, such as file replacement warnings.
    \end{itemize}
\item {\bf author:}H. Shang ANL/APS.
\end{itemize}


