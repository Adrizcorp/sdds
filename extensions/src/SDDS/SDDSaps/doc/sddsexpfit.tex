\begin{latexonly}
\newpage
\end{latexonly}
\subsection{sddsexpfit}
\label{sddsexpfit}

\begin{itemize}
\item {\bf description:}
{\tt sddsexpfit} does exponential fits to a single column of an SDDS file as a function of another column (the independent 
variable).  The fitting function is
\[ E(x) = C + F * e {R*x}, \]
where x is the independent variable, C is the {\em constant} term, F is the {\em factor}, and R is the {\rm rate}.
\item {\bf examples:} 
Fit an  exponential decay to vacuum pressure versus time during a pumpdown:
\begin{flushleft}{\tt
sddsexpfit vacDecay.sdds -columns=Time,Pressure vacDecay.fit
}\end{flushleft}
Same, but give the program a hint and force it to get a better fit
\begin{flushleft}{\tt
sddsexpfit vacDecay.sdds -columns=Time,Pressure vacDecay.fit -clue=decays -tolerance=1e-12
}\end{flushleft}
\item {\bf synopsis:} 
\begin{flushleft}{\tt
sddsexpfit [-pipe=[input][,output]] [{\em inputFile}] [{\em outputFile}]
[-columns={\em xName},{\em yName}] [-tolerance={\em value}] 
[-clue=\{grows | decays\}] [-guess={\em constant},{\em factor},{\em rate}] 
[-verbosity={\em integer}] [-fullOutput] 
}\end{flushleft}
\item {\bf files:}
{\em inputFile} contains the columns of data to be fit.  If {\em inputFile} contains multiple
pages, each page of data is fit separately.  {\em outputFile} has columns containing the
independent variable data and the corresponding values of the fit.  The name of the latter
column is constructed by appending the string {\tt Fit} to the name of the dependent variable.
In addition, if {\tt -fullOutput} is given, {\em outputFile} includes a column with the dependent values and
the residual (dependent values minus fit values).  The name of the residual column is
constructed by appending the string {\tt Residual} to the name of the dependent variable.  {\em
outputFile} contains four parameters: {\tt expfitConstant}, {\tt expfitFactor}, {\tt expfitRate},
and {\tt expfitRmsResidual}.  The first three parameters are respectively C, F, and R from the above
equation.  The last is the rms residual of the fit.
\item {\bf switches:}
    \begin{itemize}
    \item {\tt -pipe=[input][,output]} --- The standard SDDS Toolkit pipe option.
    \item {\tt -columns={\em xName},{\em yName}} --- Specifies the names of the independent and dependent columns of data.
    \item {\tt -tolerance={\em value}} --- Specifies how close {\tt sddsexpfit} will attempt to come to the optimum fit,
        in terms of the mean squared residual.  The default is ${\rm 10 {-8}}$.
    \item {\tt -clue=\{grows | decays\}} --- Helps {\tt sddsexpfit} decide whether the data is a decaying or growing exponential,
        i.e., whether R is negative or positive, respectively.  If {\tt sddsexpfit} is having trouble, this
        will often help.
    \item {\tt -guess={\em constant},{\em factor},{\em rate}} --- Gives {\tt sddsexpfit} a stating point for each of the three fit parameters.
    \item {\tt -fullOutput} --- Specifies that {\em outputFile} will contain the original dependent variable
        data and the fit residuals, in addition to the independent variable data and the fit values.
    \item {\tt -verbosity={\em integer}} --- Specifies that informational printouts are desired during fitting.  A larger
        integer produces more output.
    \end{itemize}
\item {\bf see also:}
    \begin{itemize}
    \item \hyperref{Data for Examples}{Data for Examples (see }{)}{exampleData}
    \item \progref{sddspfit}
    \item \progref{sddsgfit}
    \item \progref{sddsoutlier}
    \end{itemize}
\item {\bf author:} M. Borland, ANL/APS.
\end{itemize}


