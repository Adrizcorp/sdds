\begin{latexonly}
\newpage
\end{latexonly}
\subsection{sddssplit}
\label{sddssplit}

\begin{itemize}
\item {\bf description:} \hspace*{1mm}\\
{\tt sddssplit} breaks up an SDDS file into one or more separate files, each containing only a single
page of data.  This may be useful in those instances where a tool or program only processes the first
page of a file.
\item {\bf examples:} \\
Split a Twiss parameter file into separate files:
\begin{flushleft}{\tt
sddssplit APS.twi
}\end{flushleft}
\item {\bf synopsis:} \\
\begin{flushleft}{\tt
sddssplit \{-pipe[=input] | {\em inputFile}\}  [\{-binary | -ascii\}] 
[-digits={\em number}] [-rootname={\em string}] [-extension={\em string}]
[-nameParameter={\em paramName}]
[-firstPage={\em number}] [-lastPage={\em number}] [-interval={\em number}] 
}\end{flushleft}
\item {\bf files:}
{\em inputFile} is an SDDS file to be split.  By default, the output files are created by appending the page number
to a ``rootname'' and adding an extension.  That is, the output files have names 
{\em rootname}{\em Page}.{\em extension}.  The default rootname is the name of {\em inputFile}, while
the default extension is ``sdds''.  By default, {\em Page} is printed using ``%03ld'' format.
less the extension.  
\item {\bf switches:}
    \begin{itemize}
    \item {\tt -pipe[=input][,output]} --- The standard SDDS Toolkit pipe option.
    \item {\tt -binary}, {\tt -ascii} --- Specifies binary or ASCII output, with binary being the default.
    \item {\tt -digits={\em number}} --- Specifies the number of digits to be used in creating filenames.
        Leading zeros are included.
    \item {\tt -rootname={\em string}} --- Specifies the rootname to be used in creating filenames.
    \item {\tt -extension={\em string}} --- Specifies the extension to be used in creating filenames.
    \item {\tt -nameParameter={\em paramName}} --- Specifies that instead of composing names for the output
        files, {\tt sddssplit} take the names from the string parameter {\em paramName} in the input file.
        This provides a limited capability to retrieve the original files from a file made with {\tt sddscombine}.
        Note that if the named parameter takes the same value on two pages, the file created for the first
        of the pages will be overwritten.
    \item {\tt -firstPage={\em number}} --- Specifies the first page of data to use.
    \item {\tt -lastPage={\em number}} --- Specifies the last page of data to use.
    \item {\tt -interval={\em number}} --- Specifies the interval between pages that are used.
    \end{itemize}
\item {\bf see also:}
    \begin{itemize}
    \item \progref{sddsbreak}
    \item \progref{sddscombine}
    \end{itemize}
\item {\bf author:} M. Borland, ANL/APS.
\end{itemize}

