\begin{latexonly}
\newpage
\end{latexonly}
\subsection{sddscombine}
\label{sddscombine}

\begin{itemize}
\item {\bf description:}
\verb|sddscombine| combines data from a series of SDDS files into a single SDDS file with one page for each page in each file.  
Data is added from files in the order that they are listed on the command line.
All of the data files must contain the columns and parameters contained by the first; the program ignores
any columns or parameters in a subsequent data file that are not in the first data file.
\item {\bf example:}
Combine several Twiss parameter files into one file, keeping page boundaries separate.
\begin{flushleft}{\tt
sddscombine APS1.twi APS2.twi APS3.twi APSall.twi 
}\end{flushleft}

\item {\bf synopsis:} 
\begin{flushleft}{\tt
sddscombine [{\em inputFileList}] [{\em outputFile}] [-pipe[=input][,output]]
[-merge[={\em parameterName}]] [-overWrite] [-sparse={\em integer}] 
[-collapse]
[-delete=\{columns | parameters | arrays\},{\em matchingString}[,...]]
[-retain=\{columns | parameters | arrays\},{\em matchingString}[,...]]
}\end{flushleft}

\item {\bf files:}
{\tt {\em inputFileList}} is a list of space-seperated filenames to be combined.  {\tt {\em outputFile}} is a filename
into which the combined data is placed.  If no {\tt -pipe} options are given, the {\em outputFile} is taken as the last
filename on the commandline.  To specify an output file with input from a pipe, one uses {\tt sddscombine }{\tt
-pipe=input }{\em outputFile}.  Similarly to specify output to a pipe with many input files, use {\tt sddscombine } {\tt
-pipe=output} {\em inputFileList}.  Since accidentally leaving off the {\tt -pipe=output} option for the last command
might result in replacement of an intended input file, the program refuses to overwrite an existing file unless the {\tt
-overWrite} option is given.  A string parameter ({\tt Filename}) is included in {\tt outputFile} to show the source of
each page.
\item {\bf switches:}
    \begin{itemize}
    \item {\tt -pipe[=input][,output]} --- The standard SDDS Toolkit pipe option.
    \item {\tt -merge[={\em parameterName}]} --- 
        Specifies that all pages of all files are to be merged into a single page
        of the output file.  If a {\tt {\em parameterName}} is given, successive pages are merged only if the
        value of the named parameter is the same.
    \item \verb|overWrite| --- Forces {\tt sddscombine} to overwrite {\tt {\em outputfile}} if it exists.
    \item \verb|sparse={\em integer}| --- Specifies sparsing the tabular data in the input to retain
        only every {\em integer}-th row.
    \item \verb|collapse| --- Specifies collapsing the data, as done by \verb|sddscollapse|.
    \item {\tt -delete=\{columns | parameters | arrays\},{\em matchingString}[,{\em matchingString}...]},\\
        {\tt -retain=\{columns | parameters | arrays\},{\em matchingString}[,{\em matchingString}...]}
         --- These options specify wildcard strings to be used to select entities
        (i.e., columns, parameters, or arrays) that will respectively be deleted or retained (i.e., that will not or
        will appear in the output).   
        The selection is performed by determining which input entities have names matching any of the strings.
        If \verb|retain| is given but \verb|delete| is not, only those entities matching one of the
        strings given with \verb|retain| are retained.  If both \verb|delete| and \verb|retain|
        are given, then all entities are retained except those that match a \verb|delete| string without
        matching any of the \verb|retain| strings.
    \end{itemize}
\item {\bf see also:}
    \begin{itemize}
    \item \hyperref{Data for Examples}{Data for Examples (see }{)}{exampleData}
    \item \progref{sddscollapse}
    \end{itemize}
\item {\bf author:} M. Borland, ANL/APS.
\end{itemize}

