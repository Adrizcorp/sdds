\begin{latexonly} 
\newpage 
\end{latexonly} 
 
\subsection{sddsfdfilter} 
\label{sddsfdfilter} 
 
\begin{itemize} 
\item {\bf description:} 
 
%\item {\bf examples:}  
% 
%\begin{flushleft}{\tt  
% 
%} 
%\end{flushleft} 

\item {\bf synopsis:}  
 
\begin{flushleft}{\tt 
sddsfdfilter [-pipe[=input][,output]] [{\em inputfile}] [{\em outputfile}] 
[-columns={\em indep-variable}[,{\em depen-quantity}][,{\em depen-quantity}...]] 
[-exclude={\em depen-quantity}[,{\em depen-quantity}]] 
[-clipFrequencies=[high={\em number}][,low={\em number}]] 
[-threshold=level={\em value}[,fractional][,start={\em freq}][,end={\em freq}]]  
[-highpass=start={\em freq},end={\em freq}] [-lowpass=start={\em freq},end={\em freq}] 
[-notch=center={\em center},flatWidth={\em width1},fullWidth={\em width2}]
[-bandpass=center={\em center},flatWidth={\em width1},fullWidth={\em width2}] 
[-filterFile=filename={\em filename},frequency={\em columnName},
  \{real={\em columnName},imaginary={\em columnName} | magnitude={\em columnName}\}]}
\\ {\tt
[-cascade] [-newColumns] [-differenceColumns] 
}\end{flushleft} 

\item {\bf switches:} 
    \begin{itemize} 
    \item {\tt -pipe[=input][,output]} --- The standard SDDS Toolkit pipe option. 
    \item {\tt -columns={\em indepVariable}[,{\em depenQuantity}][,{\em depenQuantity}...] } ---  
    Gives the name of the independent variable (typically the time variable) 
    with respect to which filtering is done.  If no {\em depenQuantity} 
    qualifiers are given, then all numerical columns are filtered; 
    otherwise, only the named columns are filtered. 
   \item {\tt -exclude={\em depenQuantity}[,{\em depenQuantity}] } --- 
   Specifies the names of columns to exclude from time filtering, 
   altering whatever selections are made by the {\tt -columns} 
   option. 
    \item {\tt -clipFrequencies[=low={\em frequency}][,high={\em frequency}]} 
    --- Specifies clipping frequencies below a given frequency ({\tt low}  
    qualifier) and/or above a given frequency ({\tt high} qualifier). 
    Any frequencies in the signals that are clipped are set to zero. 
    \item {\tt -threshold=level={\em value}[,fractional][,start={\em freq}][,end={\em freq}]] }  
    --- Specifies a threshold level below which Fourier components are 
    set to zero.  If the {\tt fractional} qualifier is given, the level 
    is interpreted as a fraction of the largest component. 
    The {\tt start} and {\tt end} qualifiers allow restricting the  
    frequency range over which the threshold is applied. 
    \item {\tt -highpass=start={\em freq},end={\em freq}} --- 
    Specifies a highpass filter.  Frequency components below the {\tt start} 
    value are set to zero, while those above the {\tt end} value are 
    unaffected.  Those in between are multiplied by a value that varies 
    linearly from 0 to 1. 
    \item {\tt -lowpass=start={\em freq},end={\em freq}} --- 
    Specifies a lowpass filter.  Frequency components above the {\tt end} 
    value are set to zero, while those below the {\tt start} value are 
    unaffected.  Those in between are multiplied by a value that varies 
    linearly from 0 to 1. 
    \item {\tt -notch=center={\em center},flatWidth={\em width1},fullWidth={\em width2}} 
    --- Specifies a notch filter centered on a given frequency,  
    attenuating completely within a band {\em width1} wide.  
    Frequencies outside a band {\em width2} are unattenuated. 
    Frequencies between the two widths are attenuated by values varying 
    linearing from 0 to 1 as the frequency becomes more distant from 
    the center frequency. 
    \item {\tt -bandpass=center={\em center},flatWidth={\em width1},fullWidth={\em width2}} 
    --- Specifies a bandpass filter centered on a given frequency,  
    passing a band {\em width1} wide without attenuation. 
    Frequencies outside a band {\em width2} are completely attenuated. 
    Frequencies between the two widths are attenuated by values varying 
    linearing from 1 to 0 as the frequency becomes more distant from 
    the center frequency. 
    \item {\tt -filterFile=filename={\em filename},frequency={\em columnName},}
        {\tt \{real={\em columnName},imaginary={\em columnName} | magnitude={\em columnName}\} }
        --- Specifies a filter explicitly 
    as a function of frequency, using either a single value  
    (simple attenuation) or a real and imaginary value. 
    The name of the column containing frequency values must be 
    given with the {\tt frequency} qualifier. 
    In addition, one must either give the name of the column  
    containing the {\tt magnitude} (attenuation factor) or else 
    both the names of the columns containing  
    the {\tt real} and {\tt imaginary} components.   
    The function thus specified is interpolated linearly to 
    obtain values at any required frequencies.  Fourier components 
    at frequency beyond the range in the file are unaffected. 
    \item {\tt -cascade} --- By default, if several filters are 
    specified using the above options, their output is added.
    When the {\tt cascade} option is given, a 
    new sequence is started, with the original signal as input. 
    The output from all cascades is summed to obtain the final 
    result. For example, if one wants to have the nested effect 
    F4(F3(F2(F1(I0))))) (i.e. multiplier effect), one needs to give the command as following:
     \begin{flushleft}{\tt \bf
    sddsfdfilter <input> <output> -column=Index,... F1 -cascade F2 -cascade F3 -cascade F4 
     }\end{flushleft}

    If one wants F1(I0) + F2(I0) + F3(I0) + F4(I0) then one needs to write following command:(The illustration of output is shown in \htmladdnormallink{figure}{http://www.aps.anl.gov/Accelerator_Systems_Division/Operations_Analysis/manuals/example_files/sddsfdfilter_img1.html})
    \begin{flushleft}{\tt \bf
    sddsfdfilter <input> <output> -column=Index,... F1 F2 F3 F4 
    }\end{flushleft}

    \item {\tt -newColumns} --- Specifies that new columns be created 
    in the output file to hold the result of the filtering. 
    By default, the filtered data is placed in columns with the 
    same names as those in the input file.  Using this option, 
    both the filtered and unfiltered data will appear in the output 
    file.  The filtered data will have names of the form 
    {\em columnName}{\tt Filtered}, where {\em columnName} is the 
    name of the source column in the input file. 
    \item {\tt -differenceColumns} --- Specifies that new columns be 
    created in the output file that contain the difference between 
    each original column and the corresponding  
    filtered column.  The new columns have names of the form 
    {\em columnName}{\tt Difference}, where {\em columnName} is the 
    name of the source column in the input file. 
    \end{itemize} 

\item {\bf see also:} 
    \begin{itemize} 
    \item \progref{sddsdigfilter} 
    \item \progref{sddssmooth} 
    \end{itemize} 
\item {\bf author:} M. Borland, ANL/APS. 
\end{itemize} 
 
