% $Log: sddsvslopes.tex,v $
% Revision 1.5  1999/11/23 19:51:42  borland
% Added documentation for new programs.  Updated documentation for existing
% programs.
%
% Revision 1.4  1996/04/29 22:58:51  emery
% Minor spelling changes to make capitalization consistent.
%
% Revision 1.2  1996/04/28  23:07:23  emery
% Mentioned the automatic creation of the Index column.
%
%
% Template for making SDDS Toolkit manual entries.
%
\begin{latexonly}
\newpage
\end{latexonly}

%
% Substitute the program name for <programName>
%
\subsection{sddsvslopes}
\label{sddsvslopes}

\begin{itemize}
\item {\bf description:}
%
% Insert text of description (typicall a paragraph) here.
%
\verb|sddsvslopes| makes straight line fits of vectorized data
in the input file with respect to a selected parameter used as independent variable.
The simplest example of vectorized data is a data set with one parameter and two columns,
one string column of rootnames and one numerical column of data.
The fitting is looped over rows across all the 
data sets in the input file (using a selected parameter as the
independent vairable). The output file contains
vectorized slopes and intercepts data for each column specified in the input file.

\item {\bf examples:} 
%
% Insert text of examples in this section.  Examples should be simple and
% should be preceeded by a brief description.  Wrap the commands for each
% example in the following construct:
% 
%
The file corrector.sdds contains vectorized  beam position monitor (bpm)
readbacks as a function of corrector setting. The defined parameter is CorrectorSetpoint.
The defined columns are Rootname and x. Each row of the data set correspond to a different
bpm.
The bpm response to the corrector setpoints are calculated with
\begin{flushleft}{\tt
sddsvslopes corrector.sdds corrector.vslopes -independentVariable=CorrectorSetpoint 
  -columns=x
}\end{flushleft}
\item {\bf synopsis:} 
%
% Insert usage message here:
%
\begin{flushleft}{\tt
sddsvslopes [-pipe=[input][,output]] {\em inputFile} {\em outputFile} 
        -independentVariable={\em parametername}
        [-columns={\em listOfNames}] [-excludeColumns={\em listOfNames}] 
        [-sigma[=generate]] [-verbose]
}\end{flushleft}
\item {\bf files:}
% Describe the files that are used and produced
The input file contains the tabular data for fitting. The column Rootname must be present.

The output file contains one data set of vectorized slopes and intercept data.
The Rootname and Index columns from the input file is transfered to the output file.
In the column Index doesn't exist in the input file, then it is created in the output
file anyway.
The column names are {\em name}\verb|Slope|, and {\em name}\verb|Intercept| 
where {\em name} is the name of the fitted data.
If only one file is specified, then the input file is overwritten by the output.
A string parameter called \verb|IndenpendentVariable| is defined containing the name of the indepedent variable.

\item {\bf switches:}
%
% Describe the switches that are available
%
    \begin{itemize}
%
%   \item {\tt -pipe[=input][,output]} --- The standard SDDS Toolkit pipe option.
%
    \item {\tt  -pipe[=input][,output]} --- The standard SDDS Toolkit pipe option.
    \item {\tt  -independentVariable={\em parametername} }
        --- name of independent variable (default is the first valid column)
    \item {\tt  -columns={\em listOfNames}}   
        ---  columns to be individually paired with independentVariable for straight line fitting
    \item {\tt  -excludeColumns={\em listOfNames}}  ---    columns to exclude from fitting
    \item {\tt  -sigma[=generate]}  
        ---   calculates errors by interpreting column names 
        {\em name}\verb|Sigma| or \verb|Sigma|{\em name} as
        sigma of column {\em name}. If these columns don't exist
        then the program generates a common sigma from the residual of a first fit,
        and refits with these sigmas. If option -sigma=generate is given,
        then sigmas are generated from the residual of a first fit for all columns,
        irrespective of the presence of columns {\em name}\verb|Sigma| or \verb|Sigma|{\em name}.
    \item {\tt  -ascii }    ---         make output file in ascii mode (binary is the default)
    \item {\tt  -verbose }  ---         prints some output to stderr

    \end{itemize}
%\item {\bf see also:}
%    \begin{itemize}
%
% Insert references to other programs by duplicating this line and 
% replacing <prog> with the program to be referenced:
%
%    \item \progref{<prog>}
%    \end{itemize}
%
% Insert your name and affiliation after the '}'
%
\item {\bf author: L. Emery } ANL
\end{itemize}

