\begin{latexonly} 
\newpage 
\end{latexonly} 
 
\subsection{sddstimeconvert} 
\label{sddstimeconvert} 
 
\begin{itemize} 
\item {\bf description:} 
\verb|sddstimeconvert| does conversions between calendar time in 
terms of (for example) day, month, and year, and ``time-since-epoch''. 
The latter is the time in seconds since a system-defined reference 
time (e.g., 0:00 on January 1, 1970). 
 
\item {\bf examples:}  
Convert column data broken down as day, month, year, and hour to seconds-since-epoch:
\begin{flushleft}{\tt  
sddstimeconvert input.sdds output.sdds 
 -epoch=column,Time,year=TheYear,month=TheMonth,day=DayOfMonth,hour=HourOfDay
} 
\end{flushleft} 
where \verb|TheYear|, \verb|TheMonth|, \verb|DayOfMonth|, and \verb|HourOfDay|
are the names of the columns in the input file and \verb|Time| is the 
column to be created containing the time-since-epoch.

\item {\bf synopsis:}  
 
\begin{flushleft}{\tt 
sddstimeconvert [{\em inputFile}] [{\em outputFile}] [-pipe[=input][,output]] 
[-breakdown=\{column | parameter\},{\em timeName}[,year={\em newName}]
\hspace*{5mm}[,julianDay={\em newName}][,month={\em newName}][,day={\em newName}][,hour={\em newName}][,text={\em newName}]]
[-epoch=\{column | parameter\},{\em newName},year={\em name} \\ \hspace*{5mm}
[,julianDay={\em name} | month={\em name},day={\em name}],hour={\em name}]
}\end{flushleft} 
 
\item {\bf switches:} 
    \begin{itemize} 
    \item {\tt -pipe[=input][,output]} --- The standard SDDS Toolkit pipe option. 
    \item {\tt -breakdown=\{column | parameter\},{\em timeName}[,year={\em newName}]}
        {\tt [,julianDay={\em newName}][,month={\em newName}][,day={\em newName}][,hour={\em newName}]}
        {\tt [,text={\em newName}]} ---  
        Specifies conversion of the column or parameter data named
        {\em timeName} to year, Julian day, month, day, hour, and/or a text string.  
        {\em timeName} contains the time expressed as
        seconds-since-epoch.  Any number of these options may be given. 
    \item {\tt -epoch=\{column | parameter\},{\em newName},year={\em name},}\\ {\tt [julianDay={\em name} | month={\em name},day={\em name}],hour={\em name}} ---  
        Specifies conversion of column or parameter data given as year, Julian day or month/day, and hour
to seconds-since-epoch, with the result being placed in {\em newName}.  
    \end{itemize} 

\item {\bf notes:}
The hour data as used or created by \verb|sddstimeconvert| contains the floating-point time-of-day in hours.
That is, the minutes and seconds are folded into the hour value.

Year values must be the full four-digit year; e.g., year 99 is {\em not} 1999, but rather 99 AD.

\item {\bf author:} M. Borland, ANL/APS. 
\end{itemize} 
