\begin{latexonly}
\newpage
\end{latexonly}
\subsection{sddsfft}
\label{sddsfft}

\begin{itemize}
\item {\bf description:}
{\tt sddsfft} takes Fast Fourier Transforms of real data in columns.  It will transform any number of columns
simultaneously as a function of a single independent variable.  
Strictly speaking, the independent variable values should be equispaced; if they are not, {\tt sddsfft} uses
the average spacing.  The number of data points need not be a power of two.  Output of the magnitude only is the 
default, but phase and complex values are available.
\item {\bf examples:} 
Take the FFT of time series samples of PAR x beam-position-monitor readouts:
\begin{flushleft}{\tt
sddsfft par.bpm par.fft -column=Time,'P?P?x'
}\end{flushleft}
\item {\bf synopsis:} 
\begin{flushleft}{\tt
sddsfft [-pipe=[input][,output]] [{\em inputFile}] [{\em outputFile}]
-columns={\em indepVariable}[,{\em depenQuantityList}]
[-padWithZeroes | -truncate] [-sparse={\em integer}] 
[-window[=\{hanning | welch | parzen\}]] [-complexInput[=folded|unfolded]]
[-normalize] [-suppressAverage] [-fullOutput[=folded|unfolded]] [-psdOutput] [-inverse]
}\end{flushleft}
\item {\bf files:}
{\em inputFile} contains the data to be FFT'd.  One column from this file must be chosen as the independent
variable.   By default, all other columns are taken as dependent variables.  If {\em inputFile} contains multiple
pages, each is treated separately and is delivered to a separate page of {\em outputFile}.

{\em outputFile} contains a column {\tt f} for the frequency, along with one or more columns for each independent
variable.  By default, {\em outputFile} has one column named {\tt FFT}{\em indepName} containing the magnitude of the
FFT for each independent variable. If {\tt -fullOutput} is specified, {\em outputFile} contains additional
columns for, respectively, the phase (or argument), real part, and imaginary part of the FFT: {\tt Arg}{\em indepName},
{\tt Real}{\em indepName}, and {\tt Imag}{\em indepName}.  If power-spectral-density output is requested, then
a column {\tt PSD}{\em indepName} is also created.  

{\em outputFile} also contains two parameters, {\tt fftFrequencies} and {\tt fftFrequencySpacing}, giving
the number of frequencies and the frequency spacing, respectively.

\item {\bf switches:}
    \begin{itemize}
    \item \verb|pipe[=input][,output]| --- The standard SDDS Toolkit pipe option.
    \item {\tt -columns={\em indepVariable}[,{\em depenQuantityList}]} --- Specifies the name of the
        independent variable column.  Optionally, specifies a list of comma-separated, optionally
        wildcard-containing names of dependent quantities to be FFT'd as a function of the independent variable. 
        By default, all numerical columns except the independent column are FFT'd.
    \item {\tt -exclude={\em depenQuantity},...} --- Specifies optionally wildcarded names of columns
        to exclude from analysis.
    \item \verb|-padWithZeros| --- Specifies that the independent data should be padded with zeros to
        make the number of points equal to the nearest power of two.  In some cases, this will result in
        significantly greater speed.
    \item \verb|-truncate| --- Specifies that the data should be truncated so that the number of points is 
        the largest product of primes from 2 to 19 not greater than the original number of points.  
        In some cases, this will result in significantly greater speed.
    \item {\tt sparse={\em integer}} --- Specifies that the data should be uniformly sampled at the
        given integer interval.  While this reduces frequency span of the FFT, it may result in greater
        speed.
    \item {\tt window[=\{hanning | welch | parzen\}} --- Specifies that data windowing should be performed
        prior to taking FFT's, and optionally specifies the type of window.  The default is {\tt hanning}.
        Usually used to improve visibility of small features or accuracy of amplitudes for data that is
        not periodic in the total sampling time or a submultiple thereof.
    \item \verb|normalize| --- Specifies that FFT's will be normalized to give a maximum magnitude of 1.
    \item \verb|suppressAverage| --- Specifies that the average value of the data will be subtracted from
        every point prior to taking the FFT.  This may improve accuracy and visibility of small components.
    \item \verb|complexInput| --- Specifies that the names of the input columns are of the form Real<rootname>
        and Imag<rootname>, giving the real and imaginary part of a function to be analyzed. In this case, 
        the <depen-quantity> entries in the -columns option give the rootname, not the full quantity name.
        It has options folded and unfolded, unfolded means the input frequency space input is unfolded and
        it must have negative frequency. default is "folded", If no option is given, and if the input file
        has "SpectrumFolded" parameter, then it will be defined by this parameter.
    \item \verb|fullOutput| --- Specifies that in addition to the magnitude, the phase, real part, and imaginary
        part of each FFT will be included in the output. It also has folded and unfolded options, while the unfold
        option outputs the unfolded frequency-space (full FFT spectrum), but the folded option outputs the folded
        spectrum (half FFT).
    \item \verb|inverse| --- produce inverse fourier transform. when it is given, the output is always unfolded spectrum and
        it only works with complexInput that has imaginary data.
    \item \verb|psdOutput| --- Specifies that in in addition to ordinary FFT data, the power-spectral-densities
        will also be included in the output.  The units of the PSD are of the form ${\rm x^2/t}$, where
        x (t) represents the units of the independent (dependent) variable.  These units are conventional with
        PSDs, which are normalized to the frequency spacing so that integrating the PSD gives the signal power.
    \end{itemize}
\item {\bf How to use the folded and unfolded options} There are three cases as listed in the following:
    \begin{itemize}
    \item real input without inverse option
        \begin{table}[hbt]
        \caption{real input without inverse option}
        \begin{tabular}{|c|c|c|c|}  
        Input & \multicolumn{3}{|c|}{Output} \\ 
        Real Number & Real Number & Imaginary Number & output option \\ \hline
        N & N/2 & N/2  &folded \\ \hline
        N & N   & N  &unfolded \\ \hline
        \end{tabular}
        \label{table1}
        \end{table}
    \item complexInput without inverse option
        \begin{table}[hbt]
        \caption{complexInput without inverse option}
        \begin{tabular}{|c|c|c|c|c|c|c|}  
        \multicolumn{4}{|c|}{ComplexInput}  &\multicolumn{3}{|c|}{Output} \\ 
        Real Number & Imaginary Number & Condition & Input option & Real Number & Imaginary Number & output option \\ \hline
        N & N & & folded N & N & N & folded \\ \hline
        N & N & last imag=0 & folded & 2*(N-1) & 2*(N-1) &unfolded \\ \hline
        N & N & last imag!=0 & folded & 2*(N-1)+1) & 2*(N-1)+1 & unfolded \\ \hline
        N & N & & unfolded & N/2 & N/2 & folded \\ \hline
        N & N & & unfolded & N & N & unfolded \\ \hline
        \end{tabular}
        \label{table2}
        \end{table}
     \item with inverse option, since inverse only works with complexInptut, and inverse spectrum is always unfolded, so -fullOutput=folded will be changed to -fullOutput=unfolded with -inverse option.
        \begin{table}[hbt]
        \caption{complexInput with inverse option}
        \begin{tabular}{|c|c|c|c|c|c|} 
        \multicolumn{4}{|c|}{ComplexInput}  &\multicolumn{2}{|c|}{Inverse Output} \\
        Real Number & Imaginary Number & Condition & Input option & Real Number & Imaginary Number  \\ \hline
        N & N & last imag=0 & folded & 2*(N-1) & 2*(N-1) \\ \hline
        N & N & last imag!=0 & folded & 2*(N-1)+1) & 2*(N-1)+1\\ \hline
        N & N & & unfolded & N & N\\ \hline
        \end{tabular}
        \label{table3}
        \end{table}
     \end{itemize}
\item {\bf see also:}
    \begin{itemize}
    \item \hyperref{Data for Examples}{Data for Examples (see }{)}{exampleData}
    \item \progref{sddsdigfilter}
    \item \progref{sddsnaff}
    \end{itemize}
\item {\bf author:} M. Borland, ANL/APS.
\end{itemize}

