\begin{latexonly}
\newpage
\end{latexonly}
\subsection{col2sdds}
\label{col2sdds}

\begin{itemize}
\item {\bf description:}
Converts a file in \verb|column| self-describing format to SDDS.  This is of interest to
APS users only, some of whom still have programs that generate \verb|column|-format files.
\item {\bf synopsis:} 
\begin{flushleft}{\tt
col2sdds {\em inputFile} {\em outputFile} [-fixMplNames]
}\end{flushleft}
\item {\bf files:}
{\em inputFile} is a {\tt column}-format file, the SDDS equivalent of which is written to {\em outputFile}.
The ``auxiliary values'' of the {\tt columns} file are converted into SDDS parameters.  The {\tt column} table
is converted into SDDS tabular data, all columns begin double precision except the ``row label'', which
becomes a string column.
\item {\bf switches:}
    \begin{itemize}
    \item \verb|-fixMplNames| --- Requests that any column or parameter names in the input file that contain
        \verb|mpl| character set escape sequences be ``fixed''.  This results in simpler names.  The escape sequences
        are always retained in definition of the symbol for each column or parameter, and hence will appear on
        graphs as expected.
    \end{itemize}
\item {\bf author:} M. Borland, ANL/APS.
\end{itemize}

