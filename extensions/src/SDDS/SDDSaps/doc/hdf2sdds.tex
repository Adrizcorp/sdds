% $Log: hdf2sdds.tex,v $
% Revision 1.1  2000/09/15 23:30:52  emery
% first installation for R. Soliday.
%
%
\begin{latexonly}
\newpage
\end{latexonly}
\subsection{hdf2sdds}
\label{hdf2sdds}

\begin{itemize}
\item {\bf description:} Converts Hierarchical Data Format (HDF) to SDDS.
\item {\bf example:} 
\begin{flushleft}{\tt
hdf2sdds ./ data.hdf data.sdds -withIndex
}\end{flushleft}
\item {\bf synopsis:}
\begin{flushleft}{\tt
hdf2sdds [{\em HDF filepath}] [{\em HDF filename}] [{\em SDDSfile}]
[\{-arraysPreferred | -columnsPreferred\}]
[\{-binary | -ascii\}] [-withIndex] [-verbose]
}\end{flushleft}
\item {\bf files: }

{\em HDF filepath} is the name of the directory containing the HDF file.

{\em HDF filename} is the filename of the HDF file excluding the path.

{\em SDDSfile} is the SDDS output that is created.

\item {\bf switches:}
\begin{itemize}
    \item {\tt \{-arraysPreferred | -columnsPreferred\}} --- Requests that the output be in the form of an SDDS array or SDDS columns. By default, -columnsPreferred is used.
    \item {\tt \{-binary | -ascii\}} --- Requests that the output be binary or ASCII.
    \item {\tt -withIndex} --- An index column is added to the output file.
    \item {\tt -verbose} --- Prints out steps as the program runs.
\end{itemize}
\item {\bf author:} R. Soliday, ANL/APS.
\end{itemize}
